\newthought{It is}, then, a sound instinct which identifies freedom with power to frame purposes 
and to execute or carry into effect purposes so framed. Such freedom is in turn identical 
with self-control; for the formation of purposes and the organization of means to execute 
them are the work of intelligence. Plate once defined a slave as the person who executes 
the purposes of another, and, as has just been said, a person is also a slave who is 
enslaved to his own blind desires. There is, I think, no point in the philosophy of 
progressive education which is sounder than its emphasis upon the importance of the 
participation of the learner in the formation of the purposes which direct his activities in 
the learning process, just as there is no defect in traditional education greater than its 
failure to secure the active cooperation of the pupil in construction of the purposes 
involved in his studying. But the meaning of purposes and ends is not self-evident and 
self-explanatory. The more their educational importance is emphasized, the more 
important it is to understand what a purpose is; how it arises and how it functions in 
experience. 

A genuine purpose always starts with an impulse. Obstruction of the immediate 
execution of an impulse converts it into a desire. Nevertheless neither impulse nor desire 
is itself a purpose. A purpose is an end-view. That is, it involves foresight of the 
consequences which will result from acting upon impulse. Foresight of consequences 
involves the operation of intelligence. It demands, in the first place, observation of 
objective conditions and circumstances. For impulse and desire produce consequences 
not by themselves alone but through their interaction or co- operation with surrounding 
conditions. The impulse for such a simple action as walking is executed only in active 
conjunction with the ground on which one stands. Under ordinary circumstances, we do 
not have to pay much attention to the ground. In a ticklish situation we have to observe very carefully just what the conditions are, as in climbing a steep and rough mountain 
where no trail has been laid out. Exercise of observation is, then, one condition of 
transformation of impulse into a purpose. As in the sign by a railway crossing, we have to 
stop, look, and listen. 

But observation alone is not enough. We have to understand the significance of what 
we see, hear, and touch. This significance consists of the consequences that will result 
when what is seen is acted upon. A baby may see the brightness of a dame and be 
attracted thereby to reach for it. The significance of the flame is then not its brightness 
but its power to burn, as the consequence that will result from touching it. We can be 
aware of consequences only because of previous experiences. In cases that are familiar 
because of many prior experiences we do not have to stop to remember just what those 
experiences were. A dame comes to signify light and heat without our having expressly 
to think of previous experiences of heat and burning. But in unfamiliar cases, we cannot 
tell just what the consequences of observed conditions will be unless we go over past 
experiences in our mind, unless we reflect upon them and by seeing what is similar in 
them to those now present, go on to form a judgment of what may be expected in the 
present situation. The formation of purposes is, then, a rather complex intellectual 
operation. It involves (1) observation of surrounding conditions; (2) knowledge of what 
has happened in similar situations in the past, a knowledge obtained partly by recollection 
and partly from the in- formation, advice, and warning of those who have had a wider 
experience; and (3) judgment which puts together what is observed and what is recalled 
to see what they signify. A purpose differs from an original impulse and desire through 
its translation into a plan and method of action based upon foresight of the consequences 
of acting under given observed conditions in a certain way. \enquote{If wishes were horses, 
beggars would ride.} Desire for some thing may be intense. It may be so strong as to 
override estimation of the consequences that will follow acting upon it. Such occurrences 
do not provide the model for education. The crucial educational problem is that of procuring the postponement of immediate action upon desire until observation and judgment 
have intervened. Unless I am mistaken, this point is definitely relevant to the conduct of 
progressive schools. Overemphasis upon activity as an end, instead of upon intelligent 
activity, leads to identification of freedom with immediate execution of impulses and 
desires. This identification is justified by a confusion of impulse with purpose; although, 
as has just been said, there is no purpose unless overt action is postponed until there is 
foresight of the consequences of carrying the impulse into execution-a foresight that is 
impossible without observation, information, and judgment. Mere foresight, even if it 
takes the form of accurate prediction, is not, of course, enough. The intellectual 
anticipation, the idea of consequences, must blend with desire and impulse to acquire 
moving force. It then gives direction to what otherwise is blind, while desire gives ideas 
impetus and momentum. An idea then becomes a plan in and for an activity to be carried 
out. Suppose a man has a desire to secure a new home, say by building a house. No 
matter how strong his desire, it cannot be directly executed. The man must form an idea 
of what kind of house he wants, including the number and arrangement of rooms, etc. He 
has to draw a plan, and have blue prints and specifications made. Ah this might be an idle 
amusement for spare time unless he also took stock of his resources. He must consider 
the relation of his funds and available credit to the execution of the plan. He has to investigate available sites, their price, their nearness to his place of business, to a 
congenial neighborhood, to school facilities, and so on and so on. All of the things 
reckoned with: his ability to pay, size and needs of family, possible locations, etc., etc., 
are objective facts. They are no part of the original desire. But they have to be viewed 
and judged in order that a desire may be converted into a purpose and a purpose into a 
plan of action. 

All of us have desires, all at least who have not become so pathological that they are 
completely apathetic. These desires are the ultimate moving springs of action. A 
professional businessman wishes to succeed in his career; a general wishes to win the 
battle; a parent to have a comfortable home for his family, and to educate his children, 
and so on indefinitely. The intensity of the desire measures the strength of the efforts that 
will be put forth. But the wishes are empty castles in the air unless they are translated into 
the means by which they may be realized. The question of how soon of means takes the 
place of a projected imaginative end, and, since means are objective, they have to be 
studied and understood if a genuine purpose is to be formed. 

Traditional education tended to ignore the importance personal impulse and desire as 
moving springs. But this is no reason why progressive education should identify impulse 
and desire with purpose and thereby pass lightly over the need for careful observation, for 
wide range of information, and for judgment if students are to share in the formation of 
the purposes which activate them. In an educational scheme, the occurrence of a desire 
and impulse is not the final end. It is an occasion and a demand for the formation of a 
plan and method of activity. Such a plan, to repeat, can be formed only by study of 
conditions and by sewing all relevant information. 

The teacher's business is to see that the occasion is taken advantage of. Since freedom 
resides in the operations of intelligent observation and judgment by which a purpose is 
developed, guidance given by the teacher to the exercise of the pupils' intelligence is an 
aid to freedom, not a restriction upon it. Sometimes teachers seem to be afraid even to 
make suggestions to the members of a group as to what they should do. I have heard of 
cases in which children are surrounded with objects and materials and then left entirely to 
themselves, the teacher being loath to suggest even what might be done with the 
materials lest freedom be infringed upon. Why, then, even supply materials, since they 
are a source of some suggestion or other? But what is more important is that the 
suggestion upon which pupils act must in any case come from some- where. It is 
impossible to understand why a suggestion from one who has a larger experience and a 
wider horizon should not be at least as valid as a suggestion arising from some more or 
less accidental source. 

It is possible of course to abuse the office, and to force the activity of the young into 
channels which express the teacher's purpose rather than that of the pupils. But the way to 
avoid this danger is not for the adult to withdraw entirely. The way is, first, for the 
teacher to be intelligently aware of the capacities, needs, and past experiences of those 
under instruction, and, secondly, to allow the suggestion made to develop into a plan and 
project by means of the further suggestions contributed and organized into a whole by the 
members of the group. The plan, in other words, is a co-operative enterprise, not a 
dictation. The teacher's suggestion is not a mold for a cast-iron result but is a starting point 
to be developed into a plan through contributions from the experience of all engaged in 
the learning process. The development occurs through reciprocal give-and-take, the 
teacher taking but not being afraid also to give. The essential point is that the purpose 
grow and take shape through the process of social intelligence. 