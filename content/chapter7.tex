
\newthought{Allusion has} been made in passing a number of times to objective conditions 
involved in experience and to their function in promoting or failing to promote the 
enriched growth of further experience. By implication, these objective conditions, 
whether those of observation, of memory, of information procured from others, or of 
imagination, have been identified with the subject-matter of study and learning; or, 
speaking more generally, with the stuff of the course of study. Nothing, however, has 
been said explicitly so far about subject-matter as such. That topic will now be discussed. 
One consideration stands out clearly when education is conceived in terms of experience. 
Anything which can be called a study, whether arithmetic, history, geography, or one of 
the natural sciences, must be derived from materials which at the outset fall within the 
scope of ordinary life-experience. In this respect the newer education contrasts sharply 
with procedures which start with facts and truths that are outside the range of the 
experience of those thought, and which, therefore, have the problem of discovering ways 
and means of bringing them within experience. Undoubtedly one chief cause for the great 
success of newer methods in early elementary education has been its observance of the 
contrary principle. 

But finding the material for learning within experience is only the first step. The next 
step is the progressive development of what is already experienced into a fuller and richer 
and also more organized form, a form that gradually approximates that in which subject- 
matter is presented to the skilled, mature person. That this change is possible without 
departing from the organic connection of education with experience is shown by the fact 
that this change takes place outside of the school and apart from formal education. The 
infant, for example, begins with an environment of objects that is very restricted in space 
and time. That environment steadily expands by the momentum inherent in experience 
itself without aid from scholastic instruction. As the infant learns to reach, creep, walk, 
and talk, the intrinsic subject-matter of its experience widens and deepens. It comes into 
connection with new objects and events, which call out new powers, while the exercise of 
these powers refines and enlarges the content of its experience. Life-space and life- 
duration’s are expanded. The environment, the world of experience, constantly grows 
larger and, so to speak, thicker. The educator who receives the child at the end of this 
period has to find ways for doing consciously and deliberately what \enquote{nature}
accomplishes in the earlier years. 


It is hardly necessary to insist upon the first of the two conditions which have been 
specified. It is a cardinal precept of the newer school of education that the beginning of 
instruction shall be made with the experience learners already have; that this experience 
and the capacities that have been developed during its course provide the starting point 
for ah further learning. I am not so sure that the other condition, that of orderly 
development toward expansion and organization of subject-matter through growth of 
experience, receives as much attention. Yet the principle of continuity of educative 
experience requites that equal thought and attention be given to solution of this aspect of 
the educational problem. Undoubtedly this phase of the problem is more difficult than the 
other. Those who deal with the pre-school child, with the kindergarten child, and with the 
boy and girl of the early primary years do not have much difficulty in determining the 
range of past experience or in finding activities that connect in vital ways with it. With 
older children both factors of the problem offer increased difficulties to the educator. It is 
harder to find out the background of the experience of individuals and harder to find out 
just how the subject- matters already contained in that experience shall be directed so as 
to lead out to larger and better organized fields. 

It is a mistake to suppose that the principle of the leading on of experience to 
something different is adequately satisfied simply by giving pupils some new experiences 
any more than it is by seeing to it that they have greater still and ease in dealing with 
things with which they are already familiar. It is also essential that the new objects and 
events be related intellectually to those of earlier experiences, and this means that there 
be some advance made in conscious articulation of facts and ideas. It thus becomes the 
office of the educator to select those things within the range of existing experience that 
have the promise and potentiality of presenting new problems which by stimulating new 
ways of observation and judgment will expand the area of further experience He must 
constantly regard what is already won not as a fixed possession but as an agency and 
instrumentality for opening new fields which make new demands upon existing powers 
of observation and of intelligent use of memory. Connectedness in growth must be his 
constant watchword. The educator more than the member of any other profession is 
concerned to have a long look ahead. The physician may feel his job done when he has 
restored a patient to health. He has undoubtedly the obligation of advising him bow to 
live so as to avoid similar troubles in the future. But, after all, the conduct of his life is his own affair, not the physician's; and what is more important for the present point is that 
as far as the physician does occupy himself with instruction and advice as to the future of 
his patient he takes upon himself the function of an educator. The lawyer is occupied with 
winning a suit for his client or getting the latter out of some complication into which he 
has got himself. If it goes beyond the case presented to him he too becomes an educator. 
The educator by the very nature of his work is obliged to see his present work in terms of 
what it accomplishes, or fails to accomplish, for a future whose objects are linked with 
those of the present. 

Here, again, the problem for the progressive educator is more difficult than for the 
teacher in the traditional school. The latter had indeed to look ahead. But unless his 
personality and enthusiasm took him beyond the limits that hedged in the traditional school, he could content himself with thinking of the next examination period or the 
promotion to the next class. He could envisage the future in terms of factors that lay 
within the requirements of the school system as that conventionally existed. There is 
incumbent upon the teacher who links education and actual experience together a more 
serious and a harder business. He must be aware of the potentialities for leading students 
into new fields which belong to experiences already had, and must use this knowledge as 
his criterion for selection and arrangement of the conditions that influence their present 
experience. 

Because the studies of the traditional school consisted of subject-matter that was 
selected and arranged on the basis of the judgment of adults as to what would be useful 
for the young sometime in the future, the material to be learned was settled upon outside 
the present life -experience of the learner. In consequence, it had to do with the past; it 
was such as had proved useful to men in past ages. By reaction to an opposite extreme, as 
unfortunate as it was probably natural under the circumstances, the sound idea that 
education should derive its materials from present experience and should enable the 
learner to cope with the problems of the present and future has often been converted into 
the idea that progressive schools can to a very large extent ignore the past. If the present 
could be cut off from the past, this conclusion would be sound. But the achievements of 
the past provide the only means at command for understanding the present. Just as the 
individual has to draw in memory upon his own past to understand the conditions in 
which he individually finds himself, so the issues and problems of present social life are 
in such intimate and direct connection with the past that students cannot be prepared to 
understand either these problems or the best way of dealing with them without delving 
into their roots in the past. In other words, the sound principle that the objectives of 
learning are in the future and its immediate materials are in present experience can be 
carried into effect only in the degree that present experience is stretched, as it were, 
backward. It can expand into the future only as it is also enlarged to take in the past. 

If time permitted, discussion of the political and economic issues which the present 
generation will be compelled to face in the future would render this general statement 
definite and concrete. The nature of the issues cannot be understood save as we know 
how they came about. The institutions and customs that exist in the present and that give 
rise to present social ills and dislocations did not arise overnight. They have a long 
history behind them. Attempt to deal with them simply on the basis of what is obvious in 
the present is bound to result in adoption of superficial measures which in the end will 
only render existing problems more acute and more difficult to solve. Policies framed 
simply upon the ground of knowledge of the present cut off from the past is the 
counterpart of heedless carelessness in individual conduct. The way out of scholastic 
systems that made the past an end in itself is to make acquaintance with the past a means 
of understanding the present. Until this problem is worked out, the present clash of 
educational ideas and practices will continue. On the one hand, there win be reactionaries 
that claim that the main, if not the sole, business of education is transmission of the 
cultural heritage. On the other hand, there will be those who hold that we should ignore 
the past and deal only with the present and future. 



That up to the present time the weakest point in progressive schools is in the matter of 
selection and organization of intellectual subject-matter isr I think, inevitable under the 
circumstances. It is as inevitable as it is right and proper that they should break loose 
from the cut and dried material which formed the staple of the old education, In addition, 
the field of experience is very wide and it varies in its contents from place to place and 
from time t, time. A single course of studies for ah progressive schools is out of the 
question; it would mean abandoning the fundamental principle of connection with life- 
experiences. Moreover, progressive schools are new. They have had hardly more than , 
generation in which to develop. A certain amount of uncertainty and of laxity in choice 
and organization of subject-matter is, therefore, what was to be expected. It is no ground 
for fundamental criticism or complaint. 

It is a ground for legitimate criticism, however, when the ongoing movement of 
progressive education fails to recognize that the problem of selection and organization ~ 
subject-matter for study and learning is fundamental. Improvisation that takes advantage 
of special occasions prevents teaching and learning from being stereotyped and dead. 
But the basic material of study cannot be picked up in a cursory manner. Occasions 
which are not and cannot be foreseen are bound to arise wherever there is intellectual 
freedom. They should be utilized. But there is a decided difference between using them in 
the development of a continuing line of activity and trusting to them to provide the chief 
material of learning. 

Unless a given experience leads out into a held previously unfamiliar no problems 
arise, while problems are the stimulus to thinking. That the conditions found in present 
experience should be used as sources of problems is a characteristic which differentiates 
education based upon experience from traditional education. For in the latter, problems 
were set from outside. Nonetheless, growth depends upon the presence of difficulty to be 
overcome by the exercise of intelligence. Once more, it is part of the educator's 
responsibility to see equally to two things: First, that the problem grows out of the 
conditions of the experience being had in the present and that it is within the range of the 
capacity of students; and, secondly, that it is such that it arouses in the learner an active 
quest for information and for production of new ideas. The new facts and new ideas thus 
obtained become the ground for further experiences in which new problems me 
presented. The process is a continuous spiral The inescapable linkage of the present with 
the past is a principle whose application is not restricted to a study of history. Take 
natural science, for example. Contemporary social Life is what it is in very large measure 
because of the results of application of physical science. The experience of every child 
and youth, in the country and the city, is what it is in its present actuality became of 
appliances which utilize electricity, heat, and chemical processes. A child does not eat a 
meal that does not involve in its preparation and assimilation chemical and physiological 
principles. He does not read by artificial light or take a ride in a motor car or on a train 
without coming into contact with operations and processes which science has 
engendered. 

It is a sound educational principle that students should be introduced to scientific 
subject-matter and be initiated into its facts and laws through acquaintance with everyday social applications. Adherence to this method is Mt only the most direct avenue to 
understanding of science itself but as the pupils grow more mature it is also the surest 
road to the understanding of the economic and industrial problems of present society. For 
they are the products to a very large extent of the application of science in production and 
distribution of commodities and services, while the latter processes are the most 
important factor in determining the present relations of human beings and social groups 
to one another. It is absurd, then, to argue that processes similar to those studied in 
laboratories and institutes of research are not a part of the daily life- experience of the 
young and hence do not come within the scope of education based upon experience. That 
the immature cannot study scientific facts and principles in the way in which mature 
experts study them goes without saying. But this fact, instead of exempting the educator 
from responsibility for using present experiences so that learners may gradually be led, 
through extraction of facts and laws, to experience of a scientific order, sets one of his 
main problems. 

For if it is true that existing experience in detail and also on a wide scale is what it is 
because of the application of science, first, to processes of production and distribution of 
goods and services, and then to the relations which human beings sustain socially to one 
another, it is impossible to obtain an understanding of present social forces (without 
which they cannot be mastered and directed) apart from an education which leads 
learners into knowledge of the very same facts and principles which in their final 
organization constitute the sciences. Nor does the importance of the principle that 
learners should be led to acquaintance with scientific subject-matter cease with the 
insight thereby given into present social issues. The methods of science also point the 
way to the measures and policies by means of which a better social order can be brought 
into existence. The applications of science which have produced in large measure the 
social conditions which now exist do not exhaust the possible field of their application. 
For so far science has been applied more or less casually and under the influence of ends, 
such as private advantage and power, which are a heritage from the institutions of a pre- 
scientific age. 

We are told almost daily and from many sources that it is impossible for human beings 
to direct their common life intelligently. We are told, on one hand, that the complexity of 
human relations, domestic and international, and on the other hand, the fact that human 
beings are so largely creatures of emotion and habit, make impossible large-scale social 
planning and direction by intelligence. This view would be more credible if any 
systematic effort, beginning with early education and carried on through the continuous 
study and learning of the young, had ever been undertaken with a view to making the 
method of intelligence, exemplified in science, supreme in education. There is nothing in 
the inherent nature of habit that prevents intelligent method from becoming itself 
habitual; and there is nothing in the nature of emotion to prevent the development of 
intense emotional allegiance to the method. 

The case of science is here employed as an illustration of progressive selection of 
subject-matter resident in present experience towards organization: an organization which 
is free, not externally imposed, because it is in accord with the growth of experience itself. The utilization of subject-matter found in the present life-experience of the learner 
towards science is perhaps the best illustration that can be found of the basic principle of 
using existing experience as the means of carrying learners on to a wider, more refined, 
and better organized environing world, physical and human, than is found in the 
experiences from which educative growth sets out. Hogben's recent work Mathematics 
for the Million, shows how mathematics, if it is treated as a mirror of civilization and as 
a main agency in its progress, can contribute to- the desired goal its surely as can the 
physical sciences. The underlying ideal in any case is that of progressive organization of 
knowledge. It is with reference to organization of knowledge that we are likely to find 
Either-Or philosophies most acutely active. In practice, if not in so many words, it is 
often held that since traditional education rested upon a conception of organization of 
knowledge that was almost completely contemptuous of living present experience, 
therefore education based upon living experience should be contemptuous of the 
organization of facts and ideas. 

When a moment ago I called this organization an ideal, I meant, on the negative side, 
that the educator cannot start with knowledge already organized and proceed to lade it out 
in doses. But as an ideal the active process of organization facts and ideas is an ever- 
present educational process. No experience is educative that does not tend both to 
knowledge of more facts and entertaining of more ideas and to a better, a more orderly, 
arrangement of them. It is not true that organization is a principle foreign to experience. 
Otherwise experience would be so dispersive as to be chaotic. The experience of young 
children centers about persons and the home. Disturbance of the normal order of 
relationships in the family is now known by psychiatrists to be a fertile source of later 
mental, and : emotional troubles— a fact which testifies to the reality of this kind of 
organization. One of the great advances in early school education, in the kindergarten and 
early grades, is that it preserves the social and human center of the organization of 
experience, instead of the older violent shift of the center of gravity. But one of the 
outstanding problems of education, as of music, is modulation. In the case of education, 
modulation means movement from a social and human center toward a more objective 
intellectual scheme of organization, always hearing in mind, however, that intellectual 
organization is not an end in itself but is the means by which social relations, distinctively 
human ties and bonds, may be understood and more intelligently ordered. 

When education is based in theory and practice upon experience, it goes without 
saying that the organized subject-matter of the adult and the specialist cannot provide the 
starting point. Nevertheless, it represents the goal toward which education should 
continuously move. It is hardly necessary to say that one of the most fundamental 
principles of the scientific organization of knowledge is principle of cause-and-effect. 
The Hay in which this principle is grasped and formulated by the scientific specialist is 
certainly very different from the way in which can be approached in the experience of the 
young. But neither the relation nor grasp of its meaning is foreign to the experience of 
even the young child. When a child two or three years of age learns not to approach a 
flame too closely and yet to draw near enough a stove to get its warmth he is grasping 
and using the causal relation. There is no intelligent activity that does not conform to the requirements of the relation, and it is intelligent in the degree in which it is not only 
conformed to but consciously borne in mind. 

In the earlier forms of experience the causal relation does not offer itself in the 
abstract but in the form of the relation of means employed to ends attained; of the relation 
of means and consequences. Growth in judgment and understanding is essentially growth 
in ability to form purposes and to select and arrange means for their realization. The most 
elementary experiences of the young are filled with cases of the means-consequence 
relation. There is not a meal cooked nor a source of illumination employed that does not 
exemplify this relation. The trouble with education is not the absence of situations in 
which the causal relation is exemplified in the relation of means and consequences. 
Failure to utilize the situations so as to lead the learner on to grasp the relation in the 
given cases of experience is, however, only too common. The logician gives the names 
\enquote{analysis and synthesis} to the operations by which means are selected and organized in 
relation to a purpose. 

This principle determines the ultimate foundation for the utilization of activities in 
school. Nothing can be more absurd educationally than to make a plea for a variety of 
active occupations in the school while decrying the need for progressive organization of 
information and ideas. Intelligent activity is distinguished from aimless activity by the 
fact that it involves selection of means-analysis-out of the variety of conditions that are 
present, and their arrangement-synthesis-to reach an intended aim or purpose. That the 
more immature the learner is, the simpler must be the ends held in view and the more 
rudimentary the means employed, is obvious. But the principle of organization of activity 
in terms of some perception of the relation of consequences to means applies even with 
the very young. Otherwise an activity ceases to be educative because it is blind. With 
increased maturity, the problem of interrelation of means becomes more urgent. In the 
degree in which intelligent observation is transferred from the relation of means to ends 
to the more complex question of the relation of means to one another, the idea of cause 
and effect becomes prominent and explicit. The final justification of shops, kitchens, and 
so on in the school is not just that they afford opportunity for activity, but that they 
provide opportunity for the kind of activity or for the acquisition of mechanical skills 
which leads students to attend to the relation of means and ends, and then to 
consideration of the way things interact with one another to produce definite effects. It is 
the same in principle as the ground for laboratories in scientific research. 

Unless the problem of intellectual organization can be worked out on the ground of 
experience, reaction is sure to occur toward externally imposed methods of organization. 
There pre signs of this reaction already in evidence. We are told that our schools, old and 
new, are failing in the main task. They do not develop, it is said, the capacity for critical 
discrimination and the ability to reason. The ability to think is smothered, we are told, by 
accumulation of miscellaneous ill-digested information, and by the attempt to acquire 
forms of skill which will be immediately useful in the business and commercial world. 
We are told that these evils spring from the influence of science and from the 
magnification of present requirements at the expense of the tested cultural heritage from 
the past. It is argued that science and its method must be subordinated; that we must return to the logic of ultimate first principles expressed in the logic of Aristotle and St. 
Thomas, in order that the young may have sure anchorage in their intellectual and moral 
life, and not be at the mercy of every passing breeze that blows. 

If the method of science had ever been consistently and continuously applied 
throughout the day-by-day work of the school in all subjects, I should be more impressed 
by this emotional appeal than I am. I see at bottom but two alternatives between which 
education must choose if it is not to drift aimlessly. One of them is expressed by the 
attempt to induce educators to return to the intellectual methods and ideals that arose 
centuries before scientific method was developed. The appeal may be temporarily 
successful in a period when general insecurity, emotional and intellectual as well as 
economic, is rife. For under these conditions the desire to lean on fixed authority is 
active. Nevertheless, it is so out of touch with all the conditions of modern life that I 
believe it is folly to seek salvation in this direction. The other alternative is systematic 
utilization of scientific method as the pattern and ideal of intelligent exploration and 
exploitation of the potentialities inherent in experience. 

The problem involved comes home with peculiar force to progressive schools. Failure 
to give constant attention to development of the intellectual content of experiences and to 
obtain ever-increasing organization of facts and ideas may in the end merely strengthen 
the tendency towards a reactionary return to intellectual and moral authoritarianism. The 
present is not the time nor place for acquisition upon scientific method. But certain 
features of it are so closely connected with any educational scheme based upon 
experience that they should be noted. In the first place, the experimental method of 
science attaches more importance, not less, to ideas as ideas than do other methods. There 
is no such thing as experiment in the scientific sense unless action is directed by some 
lead- idea. The fact that the ideas employed are hypotheses, not final truths, is the reason. 
Why ideas are more jealously guarded and tested in science than anywhere else. The 
moment they are taken to be first truths in themselves there ceases to be any reason for 
scrupulous examination of them. As fixed truths they must he accepted and that is the end 
of the matter. But as hypotheses, they must be continuously tested and revised, a 
requirement that demands they be accurately formulated. 

In the second place, ideas or hypotheses are tested by the consequences, which they 
produce when they are acted upon. This fact means that the consequences of action must 
be carefully and discriminatingly observed Activity that is not checked by observation of 
what follows from it may be temporarily enjoyed. But intellectually it leads nowhere. It 
does not provide knowledge about the situations in which action occurs nor does it lead 
to clarification and expansion of ideas. 

In the third place, the method of intelligence manifested in the experimental method 
demands keeping track of ideas, activities, and observed consequences. Keeping track is 
a matter of reflective review and summarizing, in which there is both discrimination and 
record of the significant features of a developing experience. To reflect is to look back 
over what has been done so as to extract the net meanings, which are the capital stock for intelligent dealing with further experiences. It is the heart of intellectual organization and 
of the disciplined mind. 

I have been forced to speak in general and often abstract language. But what has been 
said is organically connected with the requirement that experiences in order to be 
educative must lead out into an expanding world of subject-matter, 1 subject-matter of 
facts or information and of ideas. This condition is satisfied only as the educator views 
teaching and learning as a continuous process of reconstruction of experience. This 
condition in turn can be satisfied only as the educator has a long look ahead, and views 
every present experience as a moving force in influencing what future experiences will 
be. I am aware that the emphasis I have placed upon scientific method may be 
misleading, for it may result only in calling up the special technique of laboratory 
research as that is conducted by specialists. But the meaning of the emphasis placed upon 
scientific method has little to do with specialized techniques. It means that scientific 
method is the only authentic means at our command for getting at the significance of our 
everyday experiences of the world in which we live. It means that scientific method 
provides a working pattern of the way in which and the conditions under which 
experiences are used to lead ever onward and outward. Adaptation of the method to 
individuals of various degrees of maturity is a problem for the educator, and the constant 
factors in the problem are the formation of ideas, acting upon ideas, observation of the 
conditions, which result, and organization of facts and ideas for future use. Neither the 
ideas, nor the activities, nor the observations, the organization are the same for a person 
six years old as they are for one twelve or eighteen years old, to say nothing of the adult 
scientist. But at every level there is an expanding development of experience if 
experience is educative in effect. Consequently, whatever the level of experience, we 
have no choice but either to operate in accord with the pattern it provides or else to 
neglect the place of intelligence in the development and control of a living and moving 
experience. 