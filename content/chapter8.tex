\newthought{In what I have said,} I have taken for granted the soundness of the principle 
that education in order to accomplish its ends both for the individual learner and for 
society must be based upon experience— which is always the actual life-experience of 
some individual. I have not argued for the acceptance of this principle nor attempted to 
justify it. Conservatives as well as radicals in education are profoundly discontented with 
the present educational situation taken as a whole. There is at least this much agreement 
among intelligent persons of both schools of educational thought. The educational system 
must move one way or another, either backward to the intellectual and moral standards of 
a pre-scientific age or forward to ever greater utilization of scientific method in the 
development of the possibilities of growing, expanding experience. I have but 
endeavored to point out some of the conditions, which must be satisfactorily fulfilled if 
education takes the latter course. 


For I am so confident of the potentialities of education when it is treated as 
intelligently directed development of the possibilities inherent in ordinary experience that 
I do not feel it necessary to criticize here the other route nor to advance arguments in 
favor of taking the route of experience. The only ground for anticipating failure in taking 
this path resides to my mind in the danger that experience and the experimental method 
will not be adequately conceived. There is no discipline in the world so severe as the 
discipline of experience subjected to the tests of intelligent development and direction. 
Hence the only ground I can see for even a temporary reaction against the standards, 
aims, and methods of the newer education is the failure of educators who professedly 
adopt them to be faithful to them in practice. As I have emphasized more than once, the 
road of the new education is not an easier one to follow than the old road but n more 
strenuous and difficult one. It will remain so until it has attained its majority and that 
attainment will require many years of serious co-operative work on the part of its 
adherents. The greatest danger that attends its future is, I believe, the idea that it is an 
easy way to follow, so easy that its course may be improvised, if not in an impromptu 
fashion, at least almost from day to day or from week to week. It is for this reason that 
instead of extolling its principles, I have confined myself to showing certain conditions 
which must be fulfilled if it is to have the successful career which by right belongs to it. 

I have used frequently in what precedes the words \enquote{progressive} and \enquote{new} education. 
I do not wish to close, however, without recording my firm belief that the fundamental 
issue is not of new versus old education nor of progressive against traditional education 
but a question of what anything whatever must be to be worthy of the name \textit{education}. I 
am not, I hope and believe, in favor of any ends or any methods simply because the name 
progressive may be applied to them. The basic question concerns the nature of education 
with no qualifying adjectives prefixed. What we want and need is education pure and 
simple, and we shall make surer and faster progress when we devote ourselves to finding 
out just what education is and what conditions have to be satisfied in order that education 
may be a reality and not a name or a slogan. It is for this reason alone that I have 
emphasized the need for a sound philosophy of experience. 
