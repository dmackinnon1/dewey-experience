
\newthought{I have said} that educational plans and projects, seeing education in terms of life 
experience, are thereby committed to framing and adopting an intelligent theory or, if you 
please, philosophy of experience. Otherwise they are at the mercy of every intellectual 
breeze that happens to blow. I have tried to illustrate the need for such a theory by calling 
attention to two principles, which are fundamental in the constitution of experience: the 
principles of interaction and of continuity. If, then, I am asked why I have spent so much 
time on expounding a rather abstract philosophy, it is because practical attempts to 
develop schools based upon the idea that education is found in life-experience are bound 
to exhibit inconsistencies and confusions unless they are guided by some conception of 
what experience is, and what marks oh educative experience from non-educative and 
mis-educative experience. I now come to a group of actual educational questions the 
discussion of which will, I hope, provide topics and material that are more concrete than 
the discussion up to this point. 

The two principles of continuity and interaction as criteria of the value of experience 
are so intimately connected that it is not easy to tell just what special educational problem 
to take up first. Even the convenient division into problems of subject-matter or studies 
and of methods of teaching and learning is likely to fail us in selection and organization 
of topics to discuss. Consequently, the beginning and sequence of topics is somewhat 
arbitrary. I shall commence, however, with the old question of individual freedom and 
social control and pass on to the questions that grow naturally out of it. 

It is often well in considering educational problems to get a start by temporarily 
ignoring the school and thinking of other human situations. I take it that no one would 
deny that the ordinary good citizen is as a matter of fact subject to a great deal of social 
control and that a considerable part of this control is not felt to involve restriction of 
personal freedom. Even the theoretical anarchist, whose philosophy commits him to the 
idea that state or government control is an unmitigated evil, believes that with abolition 
of the political state other forms of social control would operate: indeed, his opposition to 
govern- mental regulation springs from his belief that other and to him more normal 
modes of control would operate with abolition of the state. 

Without taking up this extreme position, let us note some examples of social control 
that operate in everyday life, and then look for the principle underlying them. Let us 
begin with the young people themselves. Children at recess or after school play games, from tag and one-old- cat to baseball and football. The games involve rules, and these 
rules order their conduct. The games do not go on haphazardly or by a succession of 
improvisations. Without rules there is no game. If disputes arise there is an umpire to 
appeal to, or discussion and a kind of arbitration are means to a decision; otherwise the 
game is broken up and comes to an end. 

There are certain fairly obvious controlling features of such situations to which I want 
to call attention. The first is that the rules are a part of the game. They ate not outside of 
it. No rules, then no game; different rules, then a different game. As long as the game 
goes on with a reasonable smoothness, the players do not feel that they are submitting to 
external imposition but that they are playing the game. In the second place at times feel 
that a decision isn't fair and be may even get angry. But he is not objecting to a rule but to 
what he claims is a violation of it, to some one-sided and unfair action. In the third place, 
the rules, and hence the conduct of the game, are fairly standardized. There are 
recognized ways of counting out, of selection of sides, as well as for positions to be 
taken, movements to be made, etc. These rules have the sanction of tradition and 
precedent. Those playing the game have seen, perhaps, professional matches and they 
want to emulate their elders. An element that is conventional is pretty strong. Usually, a 
group of youngsters change the rules by which they play only when the adult group to 
which they look for models have themselves made a change in the rules, while the change 
made by the elders is at least supposed to conduce to making the game more skillful or 
more interesting to spectators. 

Now, the general conclusion I would draw is that control of individual actions is 
effected by the whole situation in which individuals are involved, in which they share and 
of which they are co-operative or interacting parts. For even in a competitive game there 
is a certain kind of participation, of sharing in a common experience. Stated the other 
way mound, those who take part do not feel that they are bossed by an individual person 
or are being subjected to the will of some outside superior person. When violent disputes 
do arise, it is usually on the alleged ground that the umpire or a person on the other side is 
being unfair; in other words, that in such cases some individual is trying to impose his 
individual will on someone else. 

It may seem to be putting too heavy a load upon a single case to argue that this 
instance illustrates the general principle of social control of individuals without the 
violation of freedom. But if the matter were followed out through a number of cases, I 
think the conclusion that this particular instance does illustrate a general principle would 
be justified. Games are generally competitive. If we took instances of co-operative 
activities in which all members of a group take part, as for example in well-ordered 
family life in which there is mutual confidence, the point would be even clearer. In all 
such cases it is not the will or desire of any one person which establishes order but the 
moving spirit of the whole group. The control is social, but individuals are parts of a 
community, not outside of it. 

I do not mean by this that there are no occasions upon which the authority of, say, the 
parent does not have to intervene and exercise fairly direct control. But I do say that, in the first place, the number of these occasions is slight in comparison with the number of 
those in which the control is exercised by situations in which all take part. And what is 
even more important, the authority in question when exercised in a well-regulated 
household or other community group is not a manifestation of merely personal will; the 
parent or teacher exercises it as the representative and agent of the interests of the group 
as a whole. With respect to the first point, in a well ordered school the main reliance for 
control of this and that individual is upon the activities carried on and upon the situations 
in which these activities are maintained. The teacher reduces to a minimum the occasions 
in which he or she has to exercise authority in a personal way. When it is necessary, in 
the second place, to speak and act firmly, it is done in behalf of the interest of the group, 
not as an exhibition of personal power. This makes the difference between action, which 
is arbitrary, and that which is just and fair. 

Moreover, it is not necessary that the difference should be formulated in words, by 
either teacher or the young, in order to be felt in experience. The number of children who 
do not feel the difference (even if they cannot articulate it and reduce it to an intellectual 
principle) between action that is motivated by personal power and desire to dictate and 
action that is fair, because in the interest of all, is small. I should even be willing to say 
that upon the whole children are more sensitive to the signs and symptoms of this 
difference than are adults. Children learn the difference when playing with one another. 
They are willing, often too willing if anything, to take suggestions from one child and let 
him be a leader if his conduct adds to the experienced value of what they are doing, while 
they resent the attempt at dictation. Then they often withdraw and when asked why, say 
that it is because so-and-so \enquote{is too bossy.} 

I do not wish to refer to the traditional school in ways which set up a caricature in lieu 
of a picture. But I think it is fair to say that one reason the personal commands of the 
teacher so often played an undue role and a season why the order which existed was so 
much a matter of sheer obedience to the will of an adult was because the situation almost 
forced it upon the teacher. The school was not a group or community held together by 
participation in common activities. Consequently, the normal, proper conditions of 
control were lacking. Their absence was made up for, and to a considerable extent had to 
be made up for, by the direct intervention of the teacher, who, as the saying went, \enquote{kept 
order.} He kept it because order was in the teacher's keeping, instead of residing in the 
shared work being done. 

The conclusion is that in what are called the new schools, the primary source of social 
control resides in the very nature of the work done as a social enterprise in which all 
individuals have an opportunity to contribute and to which all feel a responsibility. Most 
children are naturally \enquote{sociable.} Isolation is even more irksome to them than to adults. A 
genuine community life has its ground in this natural sociability. But community life does 
not organize itself in an enduring way purely spontaneously. It requires thought and 
planning ahead. The educator is responsible for a knowledge of individuals and for a 
knowledge of subject-matter that will enable activity ties to be selected which lend 
themselves to social organization, an organization in which all individuals have an opportunity to contribute something, and in which the activities in which all participate 
are the chief carrier of control. 

I am not romantic enough about the young to suppose that every pupil will respond or 
that any child of normally strong impulses will respond on every occasion. There are 
likely to be some who, when they come to school, are already victims of injurious 
conditions outside of the school and who have become so passive and unduly docile that 
they fail to contribute. There will be others who, because of previous experience, are 
bumptious and unruly and perhaps downright rebellious. But it is certain that the general 
principle of social control cannot be predicated upon such cases. It is also true that no 
general rule can be laid down for dealing with such cases. The teacher has to deal with 
them individually. They fall into general classes, but no two are exactly alike. The 
educator has to discover as best he or she can the causes for the recalcitrant attitudes. He 
or she cannot, if the educational process is to go on, make it a question of pitting one will 
against another in order to see which is strongest, nor yet allow the unruly and non- 
participating pupils to stand permanently in the way of the educative activities of others. 
Exclusion perhaps is the only available measure at a given juncture, but it is no solution. 
For it may strengthen the very causes which have brought about the undesirable anti- 
social attitude, such as desire for attention or to show off. 

Exceptions rarely prove a rule or give a clew to what the rule should be. I would not, 
therefore, attach too much importance to these exceptional cases, although it is true at 
present that progressive schools are likely often to have more than their fair share of these 
cases, since parents may send children to such schools as a last resort. I do not think 
weakness in control when it is found in progressive schools arises in any event from these 
exceptional cases. It is much more likely to arise from failure to arrange in advance for 
the kind of work (by which I mean all kinds of activities engaged in) which will create 
situations that of themselves tend to exercise control over what this, that, and the other 
pupil does and how he does it. This failure most often goes back to lack of sufficiently 
thoughtful planning in advance. The causes for such lack are varied. The one, which is 
peculiarly important to mention in this connection, is the idea that such advance planning 
is unnecessary and even that it is inherently hostile to the legitimate freedom of those 
being instructed. 

Now, of course, it is quite possible to have preparatory planning by the teacher done in 
such a rigid and intellectually inflexible fashion that it does result in adult imposition, 
which is none the less external because executed with tact and the semblance of respect 
for individual freedom. But this kind of planning does not follow inherently from the 
principle involved. I do not know what the greater maturity of the teacher and the 
teacher's greater knowledge of the world, of subject-matters and of individuals, is for 
unless the teacher can arrange conditions that are conducive to community activity and to 
organization which exercises control over individual impulses by the mere fact that all 
are engaged in communal projects. Because the kind of advance planning heretofore 
engaged in has been so routine as to leave little room for the free play of individual 
thinking or for contributions due to distinctive individual experience, it does not follow 
that all planning must be rejected. On the contrary, there is incumbent upon the educator the duty of instituting: a much more intelligent, and consequently, more difficult, kind of 
planning. He must survey the capacities and needs of the particular set of individuals with 
whom he is dealing and must at the same time arrange the conditions which provide the 
subject-matter or content for experiences that satisfy these needs and develop these 
capacities. The planning must be flexible enough to permit free play for individuality of 
experience and yet firm enough to give direction towards continuous development of 
power. 

The present occasion is a suitable one to say something about the province and office 
of the teacher. The principle that development of experience comes about through 
interaction means that education is essentially a social process. This quality is realized in 
the degree in which individuals form a community group. It is absurd to exclude the 
teacher from membership in the group. As the most mature member of the group he has a 
peculiar responsibility for the conduct of the interactions and inter- communications 
which are the very life of the group as a community. That children are individuals whose 
freedom should be respected while the more mature person should have no freedom as an 
individual is an idea too absurd to require refutation. The tendency to exclude the teacher 
from a positive and leading share in the direction of the activities of the community of 
which he is a member is another instance of reaction from one extreme to another. When 
pupils were a class rather than a social group, the teacher necessarily acted largely from 
the outside, not as a director of processes of exchange in which all had a share. When 
education is based upon experience and educative experience is seen to be a social 
process, the situation changes radically. The teacher loses the position of external boss or 
dictator but takes on that of leader of group activities. 

In discussing the conduct of games as an example of normal social control, reference 
was made to the presence of a standardized conventional factor. The counterpart of this 
factor in school life is found in the question of manners, especially of good manners in 
the manifestations of politeness and courtesy. The more we know about customs in 
different parts of the world at different times in the history of mankind, the more we learn 
how much manners differ from place to place and time to time. This fact proves that there 
is a large conventional factor involved. But there is no group at any time or place which 
does not have some code of manners as, for example, with respect to proper ways of 
greeting other persons. The particular form a convention takes has nothing fixed and 
absolute about it. But the existence of some form of convention is not itself a convention. 
It is a uniform attendant of all social relationships. At the very least, it is the oil which 
prevents or reduces friction. 

It is possible, of course, for these social forms to become, as we say, \enquote{mere 
formalities.} They may become merely outward show with no meaning behind them. But 
the avoidance of empty ritualistic forms of social inter course does not mean the rejection 
of every formal element. It rather indicates the need for development of forms of 
intercourse that are inherently appropriate to social situations. Visitors to some 
progressive schools are shocked by the lack of manners they come across. One who 
knows the situation better is aware that to some extent their absence is due to the eager 
interest of children to go on with what they sue doing. In their eagerness they may, for example, bump into each other and into visitors with no word of apology. One might say 
that this condition is better than a display of merely external punctilio accompanying 
intellectual and emotional lack of interest in schoolwork. But it also represents a failure 
in education, a failure to learn one of the most important lessons of life, that of mutual 
accommodation and adaptation. Education is going on in a one-sided way, for attitudes 
and habits are in process of formation that stand in the way of the future learning that 
springs from easy and ready contact and communication with others. 
