\newthought{In short}, the point I am making is that rejection of the philosophy and practice of 
traditional education sets a new type of difficult educational problem for those who 
believe in the new type of education. We shall operate blindly and in confusion until we 
recognize this fact; until we thoroughly appreciate that departure from the old solves no 
problems. What is said in the following pages is, accordingly, intended to indicate some 
of the main problems with which the newer education is confronted and to suggest the 
main lines along which their solution is to be sought. I assume that amid all uncertainties 
there is one permanent frame of reference: namely, the organic connection between 
education and personal experience; or, that the new philosophy of education is committed 
to some kind of empirical and experimental philosophy. But experience and experiment 
are not self-explanatory ideas. Rather, their meaning is part of the problem to be 
explored. To know the meaning of empiricism we need to understand what experience is. 

The belief that all genuine education comes about through experience does not mean 
that all experiences are genuinely or equally educative. Experience and education cannot 
be directly equated to each other. For some experiences are miseducative. Any 
experience is miseducative that has the effect of arresting or distorting the growth of 
further experience. An experience may be such as to engender callousness; it may 
produce lack of sensitivity and of responsiveness. Then the possibilities of having richer 
experience in the future are restricted. Again, a given experience may increase a person's 
automatic skill in a particular direction and yet tend to land him in a groove or rut; the 
effect again is to narrow the field of further experience. An experience may be 
immediately enjoyable and yet promote the formation of a slack and careless attitude; this 
attitude then operates to modify the quality of subsequent experiences so as to prevent a 
person from getting out of them what they have to give. Again, experiences may be so 
disconnected from one another that, while each is agreeable or even exciting in itself, they are not linked cumulatively to one another. Energy is then dissipated and a person 
becomes scatter- brained. Each experience may be lively, vivid, and \enquote{interesting,} and yet 
their disconnectedness may artificially generate dispersive, disintegrated, centrifugal 
habits. The consequence of formation of such habits is inability to control future 
experiences. They are then taken, either by way of enjoyment or of discontent and revolt, 
just as they come. Linder such circumstances, it is idle to talk of self-control. 

Traditional education offers a plethora of examples of experiences of the kinds just 
mentioned. It is a great mistake to suppose, even tacitly, that the traditional schoolroom 
was not a place in which pupils had experiences. Yet this is tacitly assumed when 
progressive education as a plan of learning by experience is placed in sharp opposition to 
the old. The proper line of attack is that the experiences, which were had, by pupils and 
teachers alike, were largely of a wrong kind. How many students, for example, were 
rendered callous to ideas, and how many lost the impetus to learn because of the Way in 
which learning was experienced by them? How many acquired special skills by means of 
automatic drill so that their power of judgment and capacity to act intelligently in new 
situations was limited? How many came to associate the learning process with ennui and 
boredom? How many found what they did learn so foreign to the situations of life outside 
the school as to give them no power of control over the latter? How many came to 
associate books with dull drudgery, so that they were \enquote{conditioned} to all but flashy 
reading matter? 

If I ask these questions, it is not for the sake of whole sale condemnation of the old 
education. It is for quite another purpose. It is to emphasize the fact, first, that young 
people in traditional schools do have experiences; and, secondly, that the trouble is not 
the absence of experiences, but their defective and wrong character— wrong and defective 
from the standpoint of connection with further experience. The positive side of this point 
is even more important in connection with progressive education. It is not enough to 
insist upon the necessity of experience, nor even of activity in experience Everything 
depends upon the \textit{quality} of the experience, which is had. The quality of any experience 
has two aspects. There is an immediate aspect of agreeableness or disagreeableness, and 
there is its influence upon later experiences. The first is obvious and easy to judge. The 
\textit{effect} of an experience is not borne on its face. It sets a problem to the educator. It is his 
business to arrange for the kind of experiences which, while they do not repel the student, 
but rather engage his activities are, nevertheless, more than immediately enjoyable since 
they promote having desirable future experiences Just as no man lives or dies to himself, 
so no experience lives and dies to itself. Wholly independent of desire or intent every 
experience lives on in further experiences. Hence the central problem of an education 
based upon experience is to select the kind of present experiences that live fruitfully and 
creatively in subsequent experiences. 

Later, I shall discuss in more detail the principle of the continuity of experience or 
what may be called the experiential continuum. Here I wish simply to emphasize the 
importance of this principle for the philosophy of educative experience. A philosophy of 
education, like any theory, has to be stated in words, in symbols. But so far as it is more 
than verbal it is a plan for conducting education. Like any plan, it must be framed with reference to what is to be done and how it is to be done. The more definitely and 
sincerely it is held that education is a development within, by, and for experience, the 
more important it is that there shall be clear conceptions of what experience is. Unless 
experience is so conceived that the result is a plan for deciding upon subject-matter, upon 
methods of instruction and discipline, and upon material equipment and social 
organization of the school, it is wholly in the air. It is reduced to a form of words which 
may be emotionally stirring but for which any other set of words might equally well be 
substituted unless they indicate operations to be initiated and executed. Just because 
traditional education was a matter of routine in which the plans and programs were 
handed down from the past, it does not follow that progressive education is a matter of 
planless improvisation. 

The traditional school could get along without any consistently developed philosophy 
of education. About all it required in that line was a set of abstract words like culture, 
discipline, our great cultural heritage, etc., actual guidance being derived not from them 
but from custom and established routines. Just because progressive schools cannot rely 
upon established traditions and institutional habits, they must either proceed more or less 
haphazardly or be directed by ideas which, when they are made articulate and coherent, 
form a philosophy of education. Revolt against the kind of organization characteristic of 
the traditional school constitutes a demand for a kind of organization based upon ideas. I 
think that only slight acquaintance with the history of education is needed to prove that 
educational reformers and innovators alone have felt the need for a philosophy of 
education. Those who adhered to the established system needed merely a few fine- 
sounding words to justify existing practices. The real work was done by habits, which 
were so fixed as to be institutional. The lesson for progressive education is that it requires 
in an urgent degree, a degree more pressing than was incumbent upon former innovators, 
a philosophy of education based upon a philosophy of experience. 

I remarked incidentally that the philosophy in question is, to paraphrase the saying of 
Lincoln about democracy, one of education of, by, and for experience. No one of these 
words, \textit{of, by}, or \textit{for}, names anything which is self- evident. Each of them is a challenge 
to discover and put into operation a principle of order and organization, which follows 
from understanding what educative experience, signifies. 

It is, accordingly, a much more difficult task to work out the kinds of materials, of 
methods, and of social relationships that are appropriate to the new education than is the 
case with traditional education. I think many of the difficulties experienced in the conduct 
of progressive schools and many of the criticisms leveled against them arise from this 
source. The difficulties are aggravated and the criticisms are increased when it is 
supposed that the new education is somehow easier than the old. This belief is, I imagine, 
more or less current. Perhaps it illustrates again the \enquote{Either-Or} philosophy, springing from 
the idea that about all which is required is not to do what is done in traditional schools. 

I admit gladly that the new education is simpler in principle than the old. It is in 
harmony with principles of growth, while there is very much which is artificial in the old 
selection and arrangement of subjects and methods, and artificiality always leads to unnecessary complexity. But the easy and the simple are not identical. To discover what 
is really simple and to act upon the discovery is an exceedingly difficult task. After the 
artificial and complex is once institutionally established and ingrained in custom and 
routine, it is easier to walk in the paths that have been beaten than it is, after taking a new 
point of view, to work out what is practically involved in the new point of view. The old 
Ptolemaic astronomical system was more complicated with its cycles and epicycles than 
the Copernican system. But until organization of actual astronomical phenomena on the 
ground of the latter principle had been effected the easiest course was to follow the line 
of least resistance provided by the old intellectual habit. So we come back to the idea that 
a coherent theory of experience, affording positive direction to selection and organization 
of appropriate educational methods and materials, is required by the attempt to give new 
direction to the work of the schools. The process is a slow and arduous one. It is a matter 
of growth and there are many obstacles, which tend to obstruct growth and to deflect it 
into wrong lines. 

I shall have something to say later about organization. All that is needed, perhaps, at 
this point is to say that we must escape from the tendency to think of organization in 
terms of the \textit{kind} of organization, whether of content (or subject-matter), or of methods 
and social relations, that mark traditional education. I think that a good deal of the current 
opposition to the idea of organization is due to the fact that it is so hard to get away from 
the picture of the studies of the old school. The moment \enquote{organization} is mentioned 
imagination goes almost automatically to the kind of organization that is familiar, and in 
revolting against that we are led to shrink from the very idea of any organization. On the 
other hand, educational reactionaries, who are now gathering force, use the absence of 
adequate intellectual and moral organization in the newer type of school as proof not only 
of the need of organization, but to identify any and every kind of organization with that 
instituted before the rise of experimental science. Failure to develop a conception of 
organization upon the empirical and experimental basis gives reactionaries a too easy 
victory. But the fact that the empirical sciences now offer the best type of intellectual 
organization which can be found in any field shows that there is no reason why we, who 
call ourselves empiricists, should be \enquote{pushovers} in the matter of order and organization. 
