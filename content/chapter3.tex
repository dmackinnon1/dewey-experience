
\newthought{If there is} any truth in what has been said about the need of forming a theory of 
experience in order that education may be intelligently conducted upon the basis of 
experience, it is clear that the next thing in order in this discussion is to present the 
principles that are most significant in framing this theory. I shall not; therefore, apologize 
for engaging in a certain amount of philosophical analysis, which otherwise might be out 
of place. I may, however, reassure you to some degree by saying that this analysis is not 
an end in itself but is engaged in for the sake of obtaining criteria to be applied later in 
discussion of a number of concrete and, to most persons, more interesting issues. 


I have already mentioned what I called the category of continuity, or the experiential 
continuum. This principle is involved, as I pointed out, in every attempt to discriminate 
between experiences that are worth while educationally and those that are not. It may 
seem superfluous to argue that this discrimination is necessary not only in criticizing the 
traditional type of education but also in initiating and conducting a different type. 
Nevertheless, it is advisable to pursue for a little while the idea that it is necessary. One 
may safely assume, I suppose, that one thing which has recommended the progressive 
movement is that it seems more in accord with the democratic ideal to which our people 
is committed than do the procedures of the traditional school, since the latter have so 
much of the autocratic about them. Another thing which has contributed to its favorable 
reception is that its methods are humane in comparison with the harshness so often 
attending the policies of the traditional school. 

The question I would raise concerns why we prefer democratic and humane 
arrangements to those, which are autocratic and harsh. And by \enquote{why,} I mean the reason 
for preferring them, not just the causes which lead us to the preference. One cause may be 
that we have been taught not only in the schools but by the press, the pulpit, the platform, 
and our laws and law-making bodies that democracy is the best of all social institutions. 
We may have so assimilated this idea from our surroundings that it has become an 
habitual part of our mental and moral make-up. But similar causes have led other persons 
in different surroundings to widely varying conclusions— to prefer fascism, for example. 
The cause for our preference is not the same thing as the reason why we should prefer it. 

It is not my purpose here to go in detail into the reason. But I would ask a single 
question: Can we find any reason that does not ultimately come down to the belief that 
democratic social arrangements promote a better quality of human experience, one which 
is more widely accessible and enjoyed, than do non-democratic and anti-democratic 
forms of social life? Does not the principle of regard for individual freedom and for 
decency and kindliness of human relations come back in the end to the conviction that 
these things are tributary to a higher quality of experience on the part of a greater number 
than are methods of repression and coercion or force? Is it not the reason for our 
preference that we believe that mutual consultation and convictions reached through 
persuasion, make possible a better quality of experience than can otherwise be provided 
on any wide scale? 

If the answer to these questions is in the affirmative (and personally I do not see how 
we can justify our preference for democracy and humanity on any other ground), the 
ultimate reason for hospitality to progressive education, because of its reliance upon and 
use of humane methods and its kinship to democracy, goes back to the fact that 
discrimination is made between the inherent values of different experiences. So I come 
back to the principle of continuity of experience as a criterion of discrimination. 

At bottom, this principle rests upon the fact of habit, when habit is interpreted 
biologically. The basic characteristic of habit is that every experience enacted and 
undergone modifies the one who acts and undergoes, while this modification affects, 
whether we wish it or not, the quality of subsequent experiences. For it is a somewhat different person who enters into them. The principle of habit so understood obviously 
goes deeper than the ordinary conception of a habit as a more or less fixed way of doing 
things, although it includes the latter as one of its special cases. It covers the formation of 
attitudes, attitudes that are emotional and intellectual; it covers our basic sensitivities and 
ways of meeting and responding to all the conditions which we meet in living. From this 
point of view, the principle of continuity of experience means that every experience both 
takes up something from those which have gone before and modifies in some way the 
quality of those which come after. As the poet states it, 

{\small
\begin{quotation}
\ldots all experience is an arch wherethro' \\
Gleams that untraveled world, whose margin fades \\
Forever and forever when I move. 
\end{quotation}
}

So far, however, we have no ground for discrimination among experiences. For the 
principle is of universal application. There is some kind of continuity in every case. It is 
when we note the different forms in which continuity of experience operates that we get 
the basis of discriminating among experiences. I may illustrate what is meant by an 
objection, which has been brought against an idea which I once put forth— namely, that 
the educative process can be identified with growth when that is understood in terms of 
the active participle, growing. 

Growth, or growing as developing, not only physically but intellectually and morally, 
is one exemplification of the principle of continuity. The objection made is that growth 
might take many different directions: a man, for example, who starts out on a career of 
burglary may grow in that direction, and by practice may grow into a highly expert 
burglar. Hence it is argued that \enquote{growth} is not enough; we must also specify the 
direction in which growth takes place, the end towards which it tends. Before, however, 
we decide that the objection is conclusive we must analyze the case a little further. 

That a man may grow in efficiency as a burglar, as a gangster, or as a corrupt 
politician, cannot be doubted. But from the standpoint of growth as education and 
education as growth the question is whether growth in this direction promotes or retards 
growth in general. Does this form, of growth create conditions for further growth, or does 
it set up conditions that shut oh the person who has grown in this particular direction 
from the occasions, stimuli, and opportunities for continuing growth in new directions? 
What is the effect of growth in a special direction upon the attitudes and habits which 
alone open up avenues for development in other lines? I shall leave you to answer these 
questions, saying simply that when and only when development in a particular line 
conduces to continuing growth does it answer to the criterion of education as growing. 
For the conception is one that must find universal and not specialized limited application. 

I return now to the question of continuity as a criterion by which to discriminate 
between experiences which are educative and those which are mis-educative. As we have 
seen, there is some kind of continuity in any case since every experience affects for better or worse the attitudes which help decide the quality of further experiences, by setting up 
certain preference and aversion, and making it easier or harder to act for this or that end. 
Moreover, every experience influences in some degree the objective conditions under 
which further experiences are had. For example, a child who learns to speak has a new 
facility and new desire. But he has also widened the external conditions of subsequent 
learning. When he learns to read, he similarly opens up a new environment. If a person 
decides to become a teacher, lawyer, physician, or stock-broker, when he executes his 
intention he thereby necessarily determines to some extent the environment in which he 
will act in the future. He has rendered himself more sensitive and responsive to certain 
conditions, and relatively immune to those things about him that would have been stimuli 
if he had made another choice. 

But, while the principle of continuity applies in some way in every case, the quality of 
the present experience influences the way in which the principle applies. We speak of 
spoiling a child and of the spoilt child. The effect of over-indulging a child is a 
continuing one. It sets up an attitude, which operates as an automatic demand that persons 
and objects cater to his desires and caprices in the future. It makes him seek the kind of 
situation that will enable him to do what he feels like doing at the time. It renders him 
averse to and comparatively incompetent in situations, which require effort and 
perseverance in overcoming obstacles. There is no paradox in the fact that the principle 
of the continuity of experience may operate so as to leave a person arrested on a low 
plane of development, in a way, which limits later capacity for growth. 

On the other hand, if an experience arouses curiosity, strengthens initiative, and sets 
up desires and purposes that are sufficiently intense to carry a person over dead places in 
the future, continuity works in a very different way. Every experience is a moving force. 
Its value can be judged only on the ground of what it moves toward and into. The greater 
maturity of experience which should belong to the adult as educator puts him in a 
position to evaluate each experience of the young in a way in which the one having the 
less mature experience cannot do. It is then the business of the educator to see in what 
direction an experience is heading. There is no point in his being more mature if, instead 
of using his greater insight to help organize the conditions of the experience of the 
immature, he throws away his insight. Failure to take the moving force of an experience 
into account so as to judge and direct it on the ground of what it is moving into means 
disloyalty to the principle of experience itself. The disloyalty operates in two directions. 
The educator is false to the understanding that he should have obtained from his own past 
experience. He is also unfaithful to the fact that all human experience is ultimately social: 
that it involves contact and communication. The mature person, to put it in moral terms, 
has no right to withhold from the young on given occasions whatever capacity for 
sympathetic understanding his own experience has given him. 

No sooner, however, are such things said than there is a tendency to read to the other 
extreme and take what has been said as a plea for some sort of disguised imposition from 
outside. It is worth while, accordingly, to say something about the way in which the adult 
can exercise the wisdom his own wider experience gives him without imposing a merely 
external control. On one side, it is his business to be on the alert to see what attitudes and habitual tendencies are being created. In this direction he must, if he is an educator, be 
able to judge what attitudes are actually conducive to continued growth and what are 
detrimental. He must, in addition, have that sympathetic understanding ~ individuals as 
individuals which gives him an idea of what is actually going on in the minds of those 
who are learning. It is, among other things, the need for these abilities on the part of the 
parent and teacher which makes a system of education based upon living experience, 
difficult affair to conduct successfully than it is to follow the patterns of traditional 
education. 

But there is another aspect of the matter. Experience does not go on simply inside a 
person. It does go on there, for it influences the formation of attitudes of desire and 
purpose. But this is not the whole of the story. Every genuine experience has an active 
side which changes in some degree the objective conditions under which experiences are 
had. The difference between civilization and savagery, to take an example on a large 
scale, is found in the degree in which previous experiences have changed the objective 
conditions under which subsequent experiences take place. The existence of roads, of 
means of rapid movement and transportation, tools, implements, furniture, electric light 
and power, are illustrations. Destroy the external conditions of present civilized 
experience, and for a time our experience would relapse into that of barbaric peoples. 

In a word, we live from birth to death in a world of persons and things which in large 
measure is what it is because of what has been done and transmitted from previous 
human activities. When this fact is ignored, experience is treated as if it were something 
which goes on exclusively inside an individual's body and mind. It ought not to be 
necessary to say that experience does not occur in a vacuum. There are sources outside an 
individual which give rise to experience. It is constantly fed from these springs. No one 
would question that a child in a slum tenement has a different experience from that of a 
child in a cultured home; that the country lad has a different kind of experience from the 
city boy, or a boy on the seashore one different from the lad who is brought up on inland 
prairies. Ordinarily we take such facts for granted as too commonplace to record. But 
when their educational import is recognized, they indicate the second way in which the 
educator can direct the experience of the young without engaging in imposition. A 
primary responsibility of educators is that they not only be aware of the general principle 
of the shaping of actual experience by environing conditions, but that they also recognize 
in the concrete what surroundings are conducive to having experiences that lead to 
growth. Above all, they should know how to utilize the surroundings, physical and social, 
that exist so as to extract from them all that they have to contribute to building up 
experiences that are worth while. 

Traditional education did not have to face this problem; it could systematically dodge 
this responsibility. The school environment of desks, blackboards, a small schoolyard, 
was supposed to suffice. There was no demand that the teacher should become intimately 
acquainted with the conditions of the local community, physical, historical, economic, 
occupational etc., in order to utilize them as educational resources. A system of education 
based upon the necessary connection of education with experience must, on the contrary, 
if faithful to its principle, take these things constantly into account. This tax upon the educator is another reason why progressive education is more difficult to carry on than 
was ever the traditional system. 

It is possible to frame schemes of education that pretty systematically subordinate 
objective conditions to those which reside in the individuals being educated. This 
happens whenever the place and function of the teacher, of books, of apparatus and 
equipment, of everything which represents the products of the more mature experience of 
elders, is systematically subordinated to the immediate inclinations and feelings of the 
young. Every theory which assumes that importance can be attached to these objective 
factors only at the expense of imposing external control and of limiting the freedom of 
individuals rests finally upon the notion that experience is truly experience only when 
objective conditions are subordinated to what goes on within the individuals having the 
experience. 

I do not mean that it is supposed that objective conditions can be shut out. It is 
recognized that they must enter in: so much concession is made to the inescapable fact 
that we live in a world of things and persons. But I think that observation of what goes on 
in some families and some schools would disclose that some parents and some teachers 
are acting upon the idea of subordinating objective conditions to internal ones. In that 
case, it is assumed not only that the latter are primary, which in one sense they are, but 
that just as they temporarily exist they fix the whole educational process. 

Let me illustrate from the case of an infant. The needs of a baby for food, rest, and 
activity are certainly primary and decisive in one respect. Nourishment must be provided; 
provision must be made for comfortable sleep, and so on. But these facts do not mean 
that a parent shall feed the baby at any time when the baby is cross or irritable, that there 
shall not be a program of regular hours of feeding and sleeping, etc. The wise mother 
takes account of the needs of the infant but not in a way, which dispenses with her own 
responsibility for regulating the objective conditions under which the needs are satisfied. 
And if she is a wise mother in this respect, she draws upon past experiences of experts as 
well as her own for the light that these shed upon what experiences are in general most 
conducive to the normal development of infants. Instead of these conditions being 
subordinated to the immediate internal condition of the baby, they are definitely ordered 
so that a particular kind of interaction with these immediate internal states may be 
brought about. 

The word \enquote{interaction,} which has just been used, expresses the second chief principle 
for interpreting an experience in its educational function and force. It assigns equal rights 
to both factors in experience-objective and internal conditions. Any normal experience is 
an interplay of these two sets of conditions. Taken together, or in their interaction, they 
form what we call a situation. The trouble with traditional education was not that it 
emphasized the external conditions that enter into the control of the experiences but that 
it paid so little attention to the internal factors which also decide what kind of experience 
is had. It violated the principle of interaction from one side. But this violation is no 
reason why the new education should violate the principle from the other side -- except upon the basis of the extreme Either-Or educational philosophy which has been 
mentioned. 

The illustration drawn from the need for regulation of the objective conditions of a 
baby's development indicates, first, that the parent has responsibility for arranging the 
conditions under which an infant's experience of food, sleep, etc., occurs, and, secondly, 
that the responsibility is fulfilled by utilizing the funded experience of the past, as this is 
represented, say, by the advice of competent physicians and others who have made a 
special study of normal physical growth. Does it limit the freedom of the mother when 
she uses the body of knowledge thus provided to regulate the objective conditions of 
nourishment and sleep? Or does the enlargement of her intelligence in fulfilling her 
parental function widen her freedom? Doubtless if a fetish were made of the advice and 
directions so that they came to be inflexible dictates to be followed under every possible 
condition, then restriction of freedom of both parent and child would occur. But this 
restriction would also be a limitation of the intelligence that is exercised in personal 
judgment. 

In what respect does regulation of objective conditions limit the freedom of the baby? 
Some limitation is certainly placed upon its immediate movements and inclinations when 
it is put in its crib, at a time when it wants to continue playing, or does not get food at the 
moment it would like it, or when it isn't picked up and dandled when it cries for attention. 
Restriction also occurs when mother or nurse snatches a child away from an open fire 
into which it is about to fall. I shall have more to say later about freedom. Here it is 
enough to ask whether freedom is to be thought of and adjudged on the basis of relatively 
momentary incidents or whether its meaning is found in the continuity of developing 
experience. 

The statement that individuals live in a world means, in the concrete, that they live in 
a series of situations. And when it is said that they live in these situations, the meaning of 
the word \enquote{in} is different from its meaning when it is said that pennies are \enquote{in} a pocket 
or paint is \enquote{in} a can. It means, once more, that interaction is going on between an 
individual and objects and other persons. The conceptions of situation and of interaction 
are inseparable from each other. An experience is always what it is because of a 
transaction taking place between an individual and what, at the time, constitutes his 
environment, whether the latter consists of persons with whom he is talking about some 
topic or event, the subject talked about being also a part of the situation; or the toys with 
which he is playing; the book he is reading (in which his environing conditions at the 
time may be England or ancient Greece or an imaginary region); or the materials of an 
experiment he is performing. The environment, in other words, is whatever conditions 
interact with personal needs, desires, purposes, and capacities to create the experience 
which is had. Even when a person builds a castle in the air he is interacting with the 
objects which he constructs in fancy. 

The two principles of continuity and interaction are not separate from each other. 
They intercept and unite. They are, so to speak, the longitudinal and lateral aspects of 
experience. Different situations succeed one another. But because of the principle of continuity something is carried over from the earlier to the later ones. As an individual 
passes from one situation to another, his world, his environment, expands or contracts. 
He does not find himself living in another world but in a different part or aspect of one 
and the same world. What he has learned in the way of knowledge and skill in one 
situation becomes an instrument of understanding and dealing effectively with the 
situations which follow. The process goes on as long as life and learning continue. 
Otherwise the course of experience is disorderly, since the individual factor that enters 
into making an experience is split. A divided world, a world whose parts and aspects do 
not hang together, is at once a sign and a cause of a divided personality. When the 
splitting-up reaches a certain point we call the person insane. A fully integrated 
personality, on the other hand, exists only when successive experiences are integrated 
with one another. It can be built up only as a world of related objects is constructed. 

Continuity and interaction in their active union with each other provide the measure of 
the educative significance and value of an experience. The immediate and direct concern 
of an educator is then with the situations in which interaction takes place. The individual, 
who enters as a factor into it, is what he is at a given time. It is the other factor, that of 
objective conditions, which lies to some extent within the possibility of regulation by the 
educator. As has already been noted, the phrase \enquote{objective conditions} covers a wide 
range. It includes what is done by the educator and the way in which it is done, not only 
words spoken but the tone of voice in which they are spoken. It includes equipment, 
books, apparatus, toys, games played. It includes the materials with which an individual 
interacts, and, most important of all, the total social set-up of the situations in which a 
person is engaged. 

When it is said that the objective conditions are those which are within the power of 
the educator to regulate, it is meant, of course, that his ability to influence directly the 
experience of others and thereby the education they obtain places upon him the duty of 
determining that environment which will interact with the existing capacities and needs 
of those taught to create a worth-while experience. The trouble with traditional education 
was not that educators took upon themselves the responsibility for providing an 
environment. The trouble was that they did not consider the other factor in creating an 
experience; namely, the powers and purposes of those taught. It was assumed that a 
certain set of conditions was intrinsically desirable, apart from its ability to evoke a 
certain quality of response in individuals. This lack of mutual adaptation made the 
process of teaching and learning accidental. Those to whom the provided conditions were 
suitable managed to learn. Others got on as best they could. Responsibility for selecting 
objective conditions carries with it, then, the responsibility for understanding the needs 
and capacities of the individuals who are learning at a given time. It is not enough that 
certain materials and methods have proved effective with other individuals at other times. 
There must be a reason for thinking that they will function in generating an experience 
that has educative quality with particular individuals at a particular time. 

It is no reflection upon the nutritive quality of beefsteak that it is not fed to infants. It 
is not an invidious reflection upon trigonometry that we do not teach it in the first or fifth 
grade of school. It is not the subject per se that is educative or that is conducive to growth. There is no subject that is in and of itself, or without regard to the stage of 
growth attained by the learner, such that inherent educational value can be attributed to it. 
Failure to take into account adaptation to the needs and capacities of individuals was the 
source of the idea that certain subjects and certain methods are intrinsically cultural or 
intrinsically good for mental discipline. There is no such thing as educational value in the 
abstract. The notion that some subjects and methods and that acquaintance with certain 
facts and truths possess educational value in and of themselves is the reason why 
traditional education reduced the material of education so largely to a diet of predigested 
materials. According to this notion, it was enough to regulate the quantity and difficulty 
of the material provided, in a scheme of quantitative grading, from month to month and 
from year to year. Otherwise a pupil was expected to take it in doses that were prescribed 
from without. If the pupil left it in- stead of taking it, if he engaged in physical truancy, or 
in the mental truancy of mind-wandering and finally built up an emotional revulsion 
against the subject, he was held to be at fault. No question was raised as to whether the 
trouble might not lie in the subject-matter or in the way in which it was offered. The 
principle of interaction makes it clear that failure of adaptation of material to needs and 
capacities of individuals may cause an experience to be non-educative quite as much as 
failure of an individual to adapt himself to the material. 

The principle of continuity in its educational application means, nevertheless, that the 
future has to be taken into account at every stage of the educational process. This idea is 
easily misunderstood and is badly distorted in traditional education. Its assumption is, 
that by acquiring certain skills and by learning certain subjects which would be needed 
later (perhaps in college or perhaps in adult life) pupils are as a matter of course made 
ready for the needs and circumstances of the future. Now \enquote{preparation} is a treacherous
idea. In a certain sense every experience should do something to prepare a person for 
later experiences of a deeper and more expansive quality. That is the very meaning of 
growth, continuity, reconstruction of experience. But it is a mistake to suppose that the 
mere acquisition of a certain amount of arithmetic, geography, history, etc., which is 
taught and studied because it may be useful at some time in the future, has this effect, and 
it is a mistake to suppose that acquisition of skills in reading and figuring will 
automatically constitute preparation for their right and effective use under conditions 
very unlike those in which they were acquired. 

Almost everyone has had occasion to look back upon his school days and wonder 
what has become of the knowledge he was supposed to have amassed during his years of 
schooling, and why it is that the technical skills he acquired have to be learned over again 
in changed form in order to stand him in good stead. Indeed, he is lucky who does not 
find that in order to make progress, in order to go ahead intellectually, he does not have 
to unlearn much of what he learned in school. These questions cannot be disposed of by 
saying that the subjects were not actually learned for they were learned at least 
sufficiently to enable a pupil to pass examinations in them. One trouble is that the 
subject-matter in question was learned in isolation; it was put, as it were, in a water-tight 
compartment. When the question is asked, then, what has become of it, where has it gone 
to, the right answer is that it is still there in the special compartment in which it was 
originally stowed away. If exactly the same conditions recurred as those under which it was acquired, it would also recur and be available. But it was segregated when it was 
acquired and hence is so disconnected from the rest of experience that it is not available 
under the actual conditions of life. It is contrary to the laws of experience that learning of 
this kind, no matter how thoroughly engrained at the time, should give genuine 
preparation. 

Nor does failure in preparation end at this point. Perhaps the greatest of all 
pedagogical fallacies is the notion that it person learns only the particular thing he is 
studying at the time. Collateral learning in the way of formation of enduring attitudes, of 
likes and dislikes, may be and often is much more important than the spelling lesson or 
lesson in geography or history that is learned. For these attitudes are fundamentally what 
count in the future. The most important attitude that can be formed is that of desire to go 
on learning. If impetus in this direction is weakened instead of being intensified, 
something much more than mere lack of preparation takes place. The pupil is actually 
robbed of native capacities which otherwise would enable him to cope with the 
circumstances that he meets in the course of his life. We often see persons who have had 
little schooling and in whose case the absence of set schooling proves to be a positive 
asset. They have at least retained their native common sense and power of judgment, and 
its exercise in the actual conditions of living has given them the precious gift of ability to 
learn from the experiences they have. What avail is it to win pre scribed amounts of 
information about geography and history, to win ability to read and write, if in the 
process the individual loses his own soul: loses his appreciation of things worth while, of 
the values to which these things are relative; if he loses desire to apply what he has 
learned and, above all, loses the ability to extract meaning from his future experiences as 
they occur? 

What, then, is the true meaning of preparation in the educational scheme? In the first 
place, it means that a person, young or old, gets out of his present experience all that 
there is in it for him at the time in which he has it. When preparation is made the 
controlling end, then the potentialities of the present are sacrificed to a suppositious 
future. When this happens, the actual preparation for the future is missed or distorted. 
The ideal of using the present simply to get ready for the future contradicts itself. It 
omits, and even shuts out, the very conditions by which a person can be prepared for his 
future. We always live at the time we live and not at some other time, and only by 
extracting at each present time the full meaning of each present experience are we 
prepared for doing the same thing in the future. This is the only preparation which in the 
long run amounts to anything. 

All this means that attentive care must be devoted to the conditions which give each 
present experience a worthwhile meaning. Instead of inferring that it doesn't make much 
difference what the present experience is as long as it is enjoyed, the conclusion is the 
exact opposite. Here is another matter where it is easy to react from one extreme to the 
other. Because traditional schools tended to sacrifice the present to a remote and more or 
less unknown future, therefore it comes to be believed that the educator has little 
responsibility for the kind of present experiences the young undergo. But the relation of 
the present and the future is not an Either-Or affair. The present affects the future anyway. The persons who should have some idea of the connection between the two are 
those who have achieved maturity. Accordingly, upon them devolves the responsibility 
for instituting the conditions for the land of present experience which has a favorable 
effect upon the future. Education as growth or maturity should be an ever-present 
process. 

