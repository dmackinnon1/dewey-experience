\newthought{At the risk of} repeating what has been often said by me I want to say something 
about the other side of the problem of social control, namely, the nature of freedom. The 
only freedom that is of enduring importance is freedom of intelligence, that is to say, 
freedom of observation and of judgment exercised in behalf of purposes that are 
intrinsically worth while. The commonest mistake made about freedom is, I think, to 
identify it with freedom of movement, or with the external or physical side of activity. 
Now, this external and physical side of activity cannot be separated from the internal side 
of activity; from freedom of thought, desire, and purpose. The Limitation that was put 
upon outward action by the fixed arrangements of the typical traditional schoolroom, 
with its fixed rows of desks and its military regimen of pupils who were permitted to 
move only at certain fixed signals, put a great restriction upon intellectual and moral 
freedom. Straitjacket and chain-game procedures had to be done away with if there was 
to be a chance for growth of individuals in the intellectual springs of freedom without 
which there is no assurance of genuine and continued normal growth. 

But the fact still remains that an increased measure of freedom of outer movement is a 
means, not an end. The educational problem is not solved when this aspect of freedom is 
obtained. Everything then depends, so far as education is concerned, upon what is done 
with this added liberty. What end does it serve? What consequences flow from it? Let me 
speak first of the advantages which reside potentially in increase of outward freedom. In 
the first place, without its existence it is practically impossible for a teacher to gain 
knowledge of the individuals with whom he is concerned. Enforced quiet and 
acquiescence prevent pupils from disclosing their real natures. They enforce artificial 
uniformity. They put seeming before being. They place a premium upon preserving the 
outward appearance of attention, decorum, and obedience. And everyone who is 
acquainted with schools in which this system prevailed well knows that thoughts, 
imaginations, desires, and sly activities ran their own unchecked course behind this 
facade. They were disclosed to the teacher only when some untoward act led to their 
detection. One has only to contrast this highly artificial situation with normal human 
relations outside the schoolroom, say in a well conducted home, to appreciate how fatal it 
is to the teacher's acquaintance with and understanding of the individuals who are, 
supposedly, being educated. Yet without this insight there is only an accidental chance 
that the material of study and the methods used in instruction will so come home to an individual that his development of mind and character is actually directed. There is a 
vicious circle. Mechanical uniformity of studies and methods creates a kind of uniform 
immobility and this reacts to perpetuate uniformity of studies and of recitations, while 
behind this enforced uniformity individual tendencies operate in irregular and more or 
less forbidden ways. 

The other important advantage of increased outward freedom is found in the very 
nature of the learning process. That the older methods set a premium upon passivity and 
receptivity has been pointed out. Physical quiescence puts a tremendous premium upon 
these traits. The only escape from them in the standardized school is an activity, which is 
irregular and perhaps disobedient. There cannot be complete quietude in a laboratory or 
workshop. The non-social character of the traditional school is seen in the fact that it 
erected silence into one of its prime virtues. There is, of course, such a thing as intense 
intellectual activity without overt bodily activity. But capacity for such intellectual 
activity marks a comparatively late achievement when it is continued for a long period. 
There should be brief intervals of time for quiet reflection pro- vided for even the young. 
But they are periods of genuine reflection only when they follow after times of more 
overt action and are used to organize what has been gained in periods of activity in which 
the hands and other parts of the body beside the brain are used. Freedom of movement is 
also important as a means of maintaining normal physical and mental health. We have 
still to learn from the example of the Greeks who saw clearly the relation between a 
sound body and a sound mind. But in all the respects mentioned freedom of outward 
action is a means to freedom of judgment and of power to carry deliberately chosen ends 
into execution. The amount of external freedom, which is needed, varies from individual 
to individual. It naturally tends to decrease with increasing maturity, though its complete 
absence prevents even a mature individual from having the contacts, which will provide 
him with new materials upon which his intelligence may exercise itself. The amount and 
the quality of this kind of free activity as a means of growth is a problem that must 
engage the thought of the educator at every stage of development. 

There can be no greater mistake, however, than to treat such freedom as an end in 
itself. It then tends to be destructive of the shared cooperative activities which are the 
normal source of order. But, on the other hand, it turns freedom which should be positive 
into something negative. For freedom from restriction, the negative side, is to be prized 
only as a means to a freedom which is power: power to frame purposes, to judge wisely, 
to evaluate desires by the consequences which will result from acting upon them; power 
to select and order means to carry chosen ends into operation. 

Natural impulses and desires constitute in any case the starting point. But there is no 
intellectual growth without some reconstruction, some remaking, of impulses and desires 
in the form in which they first show themselves. This remaking involves inhibition of 
impulse in its first estate. The alternative to externally imposed inhibition is inhibition 
through an individual's own reflection and judgment. The old phrase \enquote{Stop and think} is  sound psychology. For thinking is stoppage of the immediate manifestation of impulse 
until that impulse has been brought into connection with other possible tendencies to 
action so that a more comprehensive and coherent plan of activity is formed. Some of the other tendencies to action lead to use of eye, ear, and hand to observe objective 
conditions; others result in recall of what has happened in the past. Thinking is thus a 
postponement of immediate action, while it effects internal control of impulse through a 
union of observation and memory, this union being the heart of reflection. What has been 
said explains the meaning of the well-worn phrase \enquote{self-control.} The ideal aim of 
education is creation of power of self-control. But the mere removal of external control is 
no guarantee for the production of self-control. It is easy to jump out of the frying-pan 
into the fire. It is easy, in other words, to escape one form of external control only to find 
oneself in another and more dangerous form of external control. Impulses and desires that 
are not ordered by intelligence are under the control of accidental circumstances. It may 
be a loss rather than a gain to escape from the control of another person only to find one's 
conduct dictated by immediate whim and caprice; that is, at the mercy of impulses into 
whose formation intelligent judgment has not entered. A person whose conduct is 
controlled in this way has at most only the illusion of freedom. Actually forces over 
which he has no command direct him. 
