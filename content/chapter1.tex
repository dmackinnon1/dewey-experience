
%Traditional vs, progressive Education 

\newthought{Mankind likes to think} in terms of extreme opposites. It is given to formulating its 
beliefs in terms of \textit{Either-Ors}, between which it recognizes no intermediate possibilities. 
When forced to recognize that the extremes cannot be acted upon, it is still inclined to 
hold that they are all right in theory but that when it comes to practical matters 
circumstances compel us to compromise. Educational philosophy is no exception. The 
history of educational theory is marked by opposition between the idea that education is 
development from within and that it is formation from without; that it is based upon 
natural endowments and that education is a process of overcoming natural inclination and 
substituting in its place habits acquired under external pressure. 

At present, the opposition, so far as practical affairs of the school are concerned, tends 
to take the form of contrast between traditional and progressive education. If the 
underlying ideas of the former are formulated broadly, without the qualification required 
for accurate statement, they are found to be about as follows: The subject-matter of 
education consists of bodies of information and of skills that have been worked out in the 
past; therefore, the chief business of the school is to transmit them to the new generation, 
In the past, there have also been developed standards and rules of conduct; moral training 
consists in forming habits of action in conformity with these rules and standards. Finally, 
the general pattern of school organization (by which I mean the relations of pupils to one 
another and to the teachers) constitutes the school kind of institution sharply marked off 
from other social institutions. Call up in imagination the ordinary school- room, its time 
schedules, schemes of classification, of examination and promotion, of rules of order, and 
I think you will grasp what is meant by \enquote{pattern of organization.} If then you contrast this 
scene with what goes on in the family, for example, you will appreciate what is meant by 
the school being a kind of institution sharply marked off from any other form of social 
organization. 

The three characteristics just mentioned fix the aims and methods of instruction and 
discipline. The main purpose or objective is to prepare the young for future 
responsibilities and for success in life, by means of acquisition of the organized bodies of 
information and prepared forms of skill, which comprehend the material of instruction. 
Since the subject-matter as well as standards of proper conduct pre handed down from the 
part, the attitude of pupils must, upon the whole, be one of docility, receptivity and 
obedience. Books, especially textbooks, are the chief representatives of the lore and 
wisdom of the past, while teachers are the organs through which pupils rue brought into 
effective connection with the material. Teachers are the agents through which knowledge 
and skills are communicated and rules of conduct: enforced 

I have not made this brief summary for the purpose of criticizing the underlying 
philosophy. The rise of what is called new education and progressive schools is of itself a 
product of discontent with traditional education. In effect it is (I criticism of the latter. 
When the implied criticism is made explicit it reads somewhat as follows: The traditional scheme is, in essence, one of imposition from above and from outside. It imposes adult 
standards, subject-matter, and methods upon those who are only growing slowly toward 
maturity. The gap is so great that the required subject-matter, the methods of learning and 
of behaving are foreign to the existing capacities of the young. They are beyond the reach 
of the experience the young learners already possess. Consequently, they must be 
imposed; even though good teachers will use devices of art to cover up the imposition so 
as to relieve it of obviously brutal features. 

But the gulf between the mature or adult products and the experience and abilities of 
the young is so wide that the very situation forbids much active participation by pupils in 
the development of what is taught. Theirs is to do— and learn, as it was the part of the six 
hundred to do and die. Learning here means acquisition of what already is incorporated in 
books and in the heads of the elders. Moreover, that which is taught is thought of as 
essentially static. It is taught as a finished product, with little regard either to the ways in 
which it was originally built up or to changes that will surely occur in the future. It is to a 
large extent the cultural product of societies that assumed the future would be much like 
the past, and yet it is used as educational food in a society where change is the rule, not 
the exception. If one attempts to formulate the philosophy of education implicit in the 
practices of the new education, we may, I think, discover certain common principles amid 
the variety of progressive schools now existing. To imposition from above is opposed 
expression and cultivation of individuality; to external discipline is opposed free activity; 
to learning from texts and teachers, learning through experience; to acquisition of isolated 
skills and techniques by drill, is opposed acquisition of them as means of attaining ends 
which make direct vital appeal; to preparation for a more or less remote future is opposed 
making the most of the opportunities of present life; to static aims and materials is 
opposed acquaintance with a changing world. 

Now, all principles by themselves are abstract. They become concrete only in the 
consequences, which result from their application. Just because the principles set forth 
are so fundamental and far-reaching, everything depends upon the interpretation given 
them as they are put into practice in the school and the home. It is at this point that the 
reference made earlier to \enquote{Either-Or} philosophies becomes peculiarly pertinent. The 
general philosophy of the new education may be sound, and yet the difference in abstract 
principles will not decide the way in which the moral and intellectual preference involved 
shall be worked out in practice. There is always the danger in a new movement that in 
rejecting the aims and methods of that which it would supplant, it may develop its 
principles negatively rather than positively and constructively. Then it takes its clew in 
practice from that which is rejected instead of from the constructive development of its 
own philosophy. 

I take it that the fundamental unity of the newer philosophy is found in the idea that 
there is an intimate and necessary relation between the processes of actual experience and 
education. If this be true, then a positive and constructive development of its own basic 
idea depends upon having a correct idea of experience. Take, for example, the question of 
organized subject-matter- which will be discussed in some detail later. The problem for 
progressive education is: What is the place and meaning of subject-matter and of organization within experience? How does subject-matter function? Is there anything 
inherent in experience, which tends towards progressive organization of its contents? 
What results follow when the materials of experience are not progressively organized? A 
philosophy which proceeds on the basis of rejection, of sheer opposition, will neglect 
these questions. It will tend to suppose that because the old education was based on 
ready-made organization, therefore it successes to reject the principle of organization in 
toto, instead of striving to discover what it means and how it is to be attained on the basis 
of experience. We might go through all the points of difference between the new and the 
old education and reach similar conclusions. When external control is rejected, the 
problem becomes that of finding the factors of control that are inherent within 
experience. When external authority is rejected, it does not follow that all authority 
should be rejected, but rather that there is need to search for a more effective source of 
authority. Because the older education imposed the knowledge, methods, and the rules of 
conduct of the mature person upon the young, it does not follow, except upon the basis of 
the extreme \enquote{Either-Or} philosophy, that the knowledge and skill of the mature person has 
no directive value for the experience of the immature. On the contrary, basing education 
upon personal experience may mean more multiplied and more intimate contacts between 
the mature and the immature than ever existed in the traditional school, and consequently 
more, rather than less, guidance by others. The problem, then, is: how these contacts can 
be established without violating the principle of learning through personal experience. 
The solution of this problem requires a well thought-out philosophy of the social factors 
that operate in the constitution of individual experience. 

What is indicated in the foregoing remarks is that the general principles of the new 
education do not of themselves solve any of the problems of the actual or practical 
conduct and management of progressive schools. Rather, they set new problems which 
have to be worked out on the basis of a new philosophy of experience. The problems are 
not even recognized, to say nothing of being solved, when it is assumed that it suffices to 
reject the ideas and practices of the old education and then go to the opposite extreme. 
Yet I am sure that you will appreciate what is meant when I say that many of the newer 
schools tend to make little or nothing of organized subject-matter of study; to proceed as 
if any form of direction and guidance by adults were an invasion of individual freedom, 
and as if the idea that education should be concerned with the present and future meant 
that acquaintance with the past has little or no role to play in education. Without pressing 
these defects to the point of exaggeration, they at least illustrate what is meant by a 
theory and practice of education which proceeds negatively or by reaction against what 
has been current in education rather than by a positive and constructive development of 
purposes, methods, and subject-matter on the foundation of a theory of experience and its 
educational potentialities. 

It is not too much to say that an educational philosophy which professes to be based 
on the idea of freedom may become as dogmatic as ever was the traditional education 
which is reacted against. For any theory and set of practices is dogmatic which is not 
based upon critical examination of its own underlying principles. Let ns say that the new 
education emphasizes the freedom of the learner. Very well. A problem is now set. What 
does freedom mean and what are the conditions under which it is capable of realization? 



Let us say that the kind of eternal imposition which was so common in the traditional 
school limited rather than promoted the intellectual and moral development of the young. 
Again, very well. Recognition of this serious defect sets a problem. Just what is the role 
of the teacher and of books in promoting the educational development of the immature 
Admit that traditional education employed as the subject-matter for study facts and ideas 
so bound up with the past as to give little help in dealing with the issues of the present 
and future. Very well. Now we have the problem of discovering the connection which 
actually exists \textit{within} experience between the achievements of the past and the issues of 
the present. We have the problem of ascertaining how acquaintance with the past may be 
translated into a potent instrumentality for dealing effectively with the future. We may 
reject knowledge of the past as the end of education and thereby only emphasize its 
importance as a means. When we do that we have a problem that is new in the story of 
education: How shall the young become acquainted with the past in such a way that the 
acquaintance is a potent agent in appreciation of the living present? 

