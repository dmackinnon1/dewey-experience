
\textit{Experience and Education} completes the first ten year cycle of Kappa Delta Pi 
Lecture series. 
% \begin{marginfigure}%
%   \includegraphics[width=\linewidth]{kappa-delta-pi-logo.jpeg}
%   %\caption{John Dewey. Born October 20, 1859 Burlington, Vermont, USA. Died June 1, 1952 (aged 92) New York City, New York, USA}
%   %\label{fig:marginfig}
% \end{marginfigure}
% \noindent
The present volume therefore is, in part, an anniversary publication 
honoring Dr. Dewey as the Society’s first and tenth lecturer. Although brief, as compared 
to the author’s other works, \textit{Experience and Education} is a major contribution to educational philosophy. Appearing in the midst of widespread confusion, which 
regrettably has scattered the forces of American education and exalted labels of conflict 
loyalties, this thin volume offers clear and certain guidance toward a united educational 
front. In as much as teachers of the \textit{new} education have avowedly applied the teachings 
of Dr. Dewey and emphasized experience, experiment, purposeful learning, freedom, and 
other well-known concepts of \textit{progressive education} it is well to learn how Dr. Dewey 
himself reacts to current and educational practices. In the interest of clear understanding 
and a union of effort the Executive Council of Kappa Delta Pi requested Dr. Dewey to 
discuss some of the moot questions that now divide American education into two camps 
and thereby weaken it at a time when its full strength is needed in guiding a bewildered 
nation through the hazards of social change. 


\textit{Experience and Education} is a lucid analysis of both \textit{traditional} and \textit{progressive} 
education. The fundamental defects of each are here described. Where the traditional 
school relied upon subjects or the cultural heritage for its content, the \textit{new} school has 
exalted the learner’s impulse and the current problems of a changing society. Neither of 
these set of values is sufficient in itself. \textit{Both} are essential. Sound educational experience 
involves, above all, continuity and interaction between the learner and what is learned. 
The traditional curriculum undoubtedly entailed rigid regimentation and a discipline that 
ignored the capacities and interests of child nature. Today, however, the reaction to this 
type of schooling often fosters the other extreme— inchoate curriculum, excessive 
individualism, and spontaneity, which is a deceptive index of freedom. Dr. Dewey insists 
that neither the old nor the new education is adequate. Each is mis-educative because 
neither of them applies the principles of a carefully developed philosophy of experience. 
Many pages of the present volume illustrate the meaning of experience and its relation to 
education. 

Frowning upon labels that express and prolong schism, Dr. Dewey interprets 
education as the scientific method by means of which man studies the world, acquires 
cumulatively knowledge of meanings and values, these outcomes, however, being data 
for critical study and intelligent living. The tendency of scientific inquiry is toward a 
body of knowledge which needs to be understood as the means whereby further inquiry 
may be directed. Hence the scientist, instead of confining his investigation to problems as 
they are discovered, proceeds to study the nature of problems, their age, conditions, 
significance. To this end he may need to review related stores of knowledge. 
Consequently, education must employ progressive organization of subject- matter in 
order that the understanding of this subject-matter may illumine the meaning and suffice 
of the problems. Scientific study leads to and enlarges experience, but this experience is 
educative only to the degree that it rests upon a continuity of significant knowledge and 
to-the degree that this knowledge modifies or \enquote{modulates} the learner's outlook, attitude, 
and skill. The true learning situation, then, has longitudinal and lateral dimensions. It is 
both historical and social. It is orderly and dynamic. Arresting pages here await the many 
educators and teachers who are earnestly seeking reliable guidance at this time. 
\textit{Experience and Education} provides a fine foundation upon which they may unitedly 
promote an American educational system which respects all sources of experience and 
rests upon a positive-not a negative- philosophy of experience and education. Directed by 
such a positive philosophy, American educators will erase their contentious labels and in 
solid ranks labor in behalf of a better tomorrow. 

\noindent
\newthought{Alfred L. Hall-Quest}\\
\noindent
Editor of Kappa Delta Pi Publications 
