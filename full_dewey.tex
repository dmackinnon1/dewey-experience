%https://archive.org/stream/ExperienceAndEducation-JohnDewey/dewey-edu-experience_djvu.txt
EXPERIENCE & EDUCATION 

John Dewey 


The great educational theorist's most concise statement of his ideas about the needs, 
the problems, and the possibilities of education— written after his experience with the 
progressive schools and in the light of the criticisms his theories received. 

"No one has done more to keep alive the fundamental ideals of liberal civilization." - 

Morris R. Cohen 

Experience and Education is the best concise statement on education ever published 
by John Dewey, the man acknowledged to be the pre-eminent educational theorist of the 
twentieth century. Written more than two decades after Democracy and Education 
(Dewey's most comprehensive statement of his position in educational philosophy), this 
book demonstrates how Dewey reformulated his ideas as a result of his intervening 
experience with the progressive schools and in the light of the criticisms his theories had 
received . 

Analyzing both "traditional" and "progressive" education, Dr. Dewey here insists that 
neither the old nor the new education is adequate and that each is miseducative because 
neither of them applies the principles of a carefully developed philosophy of experience. 
Many pages of this volume illustrate Dr. Dewey's ideas for a philosophy of experience 
and its relation to education. He particularly urges that all teachers and educators looking 
for a new movement in education should think in terms of the deeped and larger issues of 
education rather than in terms of some divisive "ism" about education, even such an 
"ism" as "progressivism." His philosophy, here expressed in its most essential, most 
readable form, predicates an American educational system that respects all sources of 
experience, on that offers a true learning situation that is both historical and social, both 
orderly and dynamic. 

"John Dewey is to be classed among those who have made philosophic thought 
relevant to the needs of their own day. In the performance of this function he is to be 
classed with the ancient stoics, with Augustine, with Aquinas, with Francis Bacon, with 
Descartes, with Locke, with Auguste Comte. " 


—Alfred North Whitehead 

"No one who is informed in the educational held can doubt for a moment the profound 
influence of John Dewey on both the theory and the practice of American education." 


—William Heard Kilpatrick 


"John Dewey is unquestionably the preeminent figure in American philosophy; no one 
has done more to keep alive the fundamental ideals of liberal civilization; and if there 
could be such an office as that of national philosopher , no one else could be properly 
mentioned for it. " 


—"Morris R. Cohen 


Preface 

ALL SOCIAL movements involve conflicts, which are reflected intellectually in 
controversies. It would not be a sign of health if such an important social interest as 
education were not also an arena of struggles, practical and theoretical. But for theory, at 
least for the theory that forms a philosophy of education, the practical conflicts and the 
controversies that are conducted upon the level of these conflicts, only set a problem. It is 
the business of an intelligent theory of education to ascertain the causes for the conflicts 
that exist and then, instead of taking one side or the other, to indicate a plan of operations 
proceeding from a level deeper and more inclusive than is represented by the practices 
and ideas of the contending parties. 

This formulation of the business of the philosophy of education does not mean that the 
latter should attempt to bring about a compromise between opposed schools of thought to 
find a via media, nor yet make an eclectic combination of points picked out hither and 
yon from all schools. It means the necessity of the introduction of a new order of 
conceptions leading to new modes of practice. It is for this mason that it is so difficult to 
develop a philosophy of education, the moment tradition and custom are departed from. It 
is for this reason that the conduct of schools, based upon a new order of conceptions, is 
so much more difficult than is the management of schools which walk in beaten paths. 
Hence, every movement in the direction of a new order of ideas and of activities directed 
by them calls out, sooner or later, a return to and practices of the past— as is exemplified 
at present in education in the attempt to revive the principles of ancient Greece and of the 
middle ages. 

It is in this context that I have suggested at the close of this little volume that those 
who are looking ahead to a new movement in education, adapted to the existing need for 
a new social order, should think in terms of Education itself rather than in terms of some 
'ism about education, even such an 'ism as "progressivism" For in spite of itself any 
movement that thinks and acts in terms of an 'ism becomes so involved in reaction 
against other 'isms that it is unwittingly controlled by them. For it then forms its 
principles by reaction against them instead of by a comprehensive, constructive survey of 
actual needs, problems, and possibilities. Whatever value is possessed by the essay 
presented in this little volume resides in its attempt to call attention to the larger and 
deeper issues of Education so as to suggest their proper frame of reference. 


John Dewey 


JOHN DEWEY, probably the most influential of all American philosophers, was born 
in Vermont in 1859. After graduation from the University of Vermont, he received a 
Ph.D. from The Johns Hopkins University and taught at a number of major universities, 
including the University of Chicago and Columbia. Before his death in 1952 he bad 
gained an international reputation for his pragmatic approach to philosophy, psychology, 
and liberal politics. 

Among his important books in these areas are: How We Think (1910), Reconstruction 
in Philosophy (1920), Experience and Nature (1925), and Logic The Theory of inquiry 
(1938). The commission, which he headed, that investigated the Moscow trials of 1936- 
37 is not example of the practical approach to political action which characterized him 
throughout his life and made him a controversial figure among liberals (though 
universally condemned by Communists). 

In all likelihood, Dewey's most enduring influence is in the field of education. 
Believing in the unity of theory and practice, Dewey not only wrote on the subject, but 
for a time participated in the "laboratory school" for children connected with the 
University of Chicago. His chief early work in this field, Democracy and Education 
(1916), was the most comprehensive statement of his position. The present work, written 
more than two decades later, shows how Dewey reformulated his ideas as a result of the 
intervening experience of the progressive schools and in the light of the criticisms his 
theories had received. Consequently, it represents the best concise statement on education 
by the most important educational theorist of the twentieth century. Moreover, it is 
probably the simplest and most readable extended statement on this subject that Dewey 
ever made. 


Editorial Foreword 

Experience and Education completes the first ten year cycle of Kappa Delta Pi 
Lecture series. The present volume therefore is, in part, an anniversary publication 
honoring Dr. Dewey as the Society’s first and tenth lecturer. Although brief, as compared 
to the author’s other works, Experience & Education is a major contribution to 
educational philosophy. Appearing in the midst of widespread confusion, which 
regrettably has scattered the forces of American education and exalted labels of conflict 
loyalties, this thin volume offers clear and certain guidance toward a united educational 
front. In as much as teachers of the “new” education have avowedly applied the teachings 
of Dr. Dewey and emphasized experience, experiment, purposeful learning, freedom, and 
other well-known concepts of “progressive education” it is well to learn how Dr. Dewey 
himself reacts to current and educational practices. In the interest of clear understanding 
and a union of effort the Executive Council of Kappa Delta Pi requested Dr. Dewey to 
discuss some of the moot questions that now divide American education into two camps 
and thereby weaken it at a time when its full strength is needed in guiding a bewildered 
nation through the hazards of social change. 


Experience & Education is a lucid analysis of both “traditional” and “progressive” 
education. The fundamental defects of each are here described. Where the traditional 
school relied upon subjects or the cultural heritage for its content, the “new” school has 
exalted the learner’s impulse and the current problems of a changing society. Neither of 
these set of values is sufficient in itself. Both are essential. Sound educational experience 
involves, above all, continuity and interaction between the learner and what is learned. 
The traditional curriculum undoubtedly entailed rigid regimentation and a discipline that 
ignored the capacities and interests of child nature. Today, however, the reaction to this 
type of schooling often fosters the other extreme— inchoate curriculum, excessive 
individualism, and spontaneity, which is a deceptive index of freedom. Dr. Dewey insists 
that neither the old nor the new education is adequate. Each is mis-educative because 
neither of them applies the principles of a carefully developed philosophy of experience. 
Many pages of the present volume illustrate the meaning of experience and its relation to 
education. 

Frowning upon labels that express and prolong schism, Dr. Dewey interprets 
education as the scientific method by means of which man studies the world, acquires 
cumulatively knowledge of meanings and values, these outcomes, however, being data 
for critical study and intelligent living. The tendency of scientific inquiry is toward a 
body of knowledge which needs to be understood as the means whereby further inquiry 
may be directed. Hence the scientist, instead of confining his investigation to problems as 
they are discovered, proceeds to study the nature of problems, their age, conditions, 
significance. To this end he may need to review related stores of knowledge. 
Consequently, education must employ progressive organization of subject- matter in 
order that the understanding of this subject-matter may illumine the meaning and suffice 
of the problems. Scientific study leads to and enlarges experience, but this experience is 
educative only to the degree that it rests upon a continuity of significant knowledge and 
to-the degree that this knowledge modifies or "modulates" the learner's outlook, attitude, 
and skill. The true learning situation, then, has longitudinal and lateral dimensions. It is 
both historical and social. It is orderly and dynamic. Arresting pages here await the many 
educators and teachers who are earnestly seeking reliable guidance at this time. 
Experience and Education provides a fine foundation upon which they may unitedly 
promote an American educational system which respects all sources of experience and 
rests upon a positive-not a negative- philosophy of experience and education. Directed by 
such a positive philosophy, American educators will erase their contentious labels and in 
solid ranks labor in behalf of a better tomorrow. 


ALFRED L. HALL-QUEST, 


Editor of Kappa Delta Pi Publications 


Chapter 1 


Traditional vs, progressive Education 

MANKIND likes to think in terms of extreme opposites. It is given to formulating its 
beliefs in terms of Either-Or, between which it recognizes no intermediate possibilities. 
When forced to recognize that the extremes cannot be acted upon, it is still inclined to 
hold that they are all right in theory but that when it comes to practical matters 
circumstances compel us to compromise. Educational philosophy is no exception. The 
history of educational theory is marked by opposition between the idea that education is 
development from within and that it is formation from without; that it is based upon 
natural endowments and that education is a process of overcoming natural inclination and 
substituting in its place habits acquired under external pressure. 

At present, the opposition, so far as practical affairs of the school are concerned, tends 
to take the form of contrast between traditional and progressive education. If the 
underlying ideas of the former are formulated broadly, without the qualification required 
for accurate statement, they are found to be about as follows: The subject- matter of 
education consists of bodies of information and of skills that have been worked out in the 
past; therefore, the chief business of the school is to transmit them to the new generation, 
In the past, there have also been developed standards and rules of conduct; moral training 
consists in forming habits of action in conformity with these rules and standards. Finally, 
the general pattern of school organization (by which I mean the relations of pupils to one 
another and to the teachers) constitutes the school kind of institution sharply marked off 
from other social institutions. Call up in imagination the ordinary school- room, its time 
schedules, schemes of classification, of examination and promotion, of rules of order, and 
I think you will grasp what is meant by "pattern of organization." If then you contrast this 
scene with what goes on in the family, for example, you will appreciate what is meant by 
the school being a kind of institution sharply marked off from any other form of social 
organization. 

The three characteristics just mentioned fix the aims and methods of instruction and 
discipline. The main purpose or objective is to prepare the young for future 
responsibilities and for success in life, by means of acquisition of the organized bodies of 
information and prepared forms of skill, which comprehend the material of instruction. 
Since the subject matter as well as standards of proper conduct pre handed down from the 
part, the attitude of pupils must, upon the whole, be one of docility, receptivity and 
obedience. Books, especially textbooks, are the chief representatives of the lore and 
wisdom of the past, while teachers are the organs through which pupils rue brought into 
effective connection with the material. Teachers are the agents through which knowledge 
and skills are communicated and rules of conduct: enforced 

I have not made this brief summary for the purpose of criticizing the underlying 
philosophy. The rise of what is called new education and progressive schools is of itself a 
product of discontent with traditional education. In effect it is (I criticism of the latter. 
When the implied criticism is made explicit it reads somewhat as follows: The traditional 


scheme is, in essence, one of imposition from above and from outside. It imposes adult 
standards, subject-matter, and methods upon those who are only growing slowly toward 
maturity. The gap is so great that the required subject matter, the methods of learning and 
of behaving are foreign to the existing capacities of the young. They are beyond the reach 
of the experience the young learners already possess. Consequently, they must be 
imposed; even though good teachers will use devices of art to cover up the imposition so 
as to relieve it of obviously brutal features. 

But the gulf between the mature or adult products and the experience and abilities of 
the young is so wide that the very situation forbids much active participation by pupils in 
the development of what is taught. Theirs is to do— and learn, as it was the part of the six 
hundred to do and die. Learning here means acquisition of what already is incorporated in 
books and in the heads of the elders. Moreover, that which is taught is thought of as 
essentially static. It is taught as a finished product, with little regard either to the ways in 
which it was originally built up or to changes that will surely occur in the future. It is to a 
large extent the cultural product of societies that assumed the future would be much like 
the past, and yet it is used as educational food in a society where change is the rule, not 
the exception. If one attempts to formulate the philosophy of education implicit in the 
practices of the new education, we may, I think, discover certain common principles amid 
the variety of progressive schools now existing. To imposition from above is opposed 
expression and cultivation of individuality; to external discipline is opposed free activity; 
to learning from texts and teachers, learning through experience; to acquisition of isolated 
skills and techniques by drill, is opposed acquisition of them as means of attaining ends 
which make direct vital appeal; to preparation for a more or less remote future is opposed 
making the most of the opportunities of present life; to static aims and materials is 
opposed acquaintance with a changing world. 

Now, all principles by themselves are abstract. They become concrete only in the 
consequences, which result from their application. Just because the principles set forth 
are so fundamental and far-reaching, everything depends upon the interpretation given 
them as they are put into practice in the school and the home. It is at this point that the 
reference made earlier to Either-Or philosophies becomes peculiarly pertinent. The 
general philosophy of the new education may be sound, and yet the difference in abstract 
principles will not decide the way in which the moral and intellectual preference involved 
shall be worked out in practice. There is always the danger in a new movement that in 
rejecting the aims and methods of that which it would supplant, it may develop its 
principles negatively rather than positively and constructively. Then it takes its clew in 
practice from that which is rejected instead of from the constructive development of its 
own philosophy. 

I take it that the fundamental unity of the newer philosophy is found in the idea that 
there is an intimate and necessary relation between the processes of actual experience and 
education. If this be true, then a positive and constructive development of its own basic 
idea depends upon having a correct idea of experience. Take, for example, the question of 
organized subject-matter- which will be discussed in some detail later. The problem for 
progressive education is: What is the place and meaning of subject-matter and of 



organization within experience? How does subject-matter function? Is there anything 
inherent in experience, which tends towards progressive organization of its contents? 
What results follow when the materials of experience are not progressively organized? A 
philosophy which proceeds on the basis of rejection, of sheer opposition, will neglect 
these questions. It will tend to suppose that because the old education was based on 
ready-made organization, therefore it successes to reject the principle of organization in 
toto, instead of striving to discover what it means and how it is to be attained on the basis 
of experience. We might go through all the points of difference between the new and the 
old education and reach similar conclusions. When external control is rejected, the 
problem becomes that of finding the factors of control that are inherent within 
experience. When external authority is rejected, it does not follow that all authority 
should be rejected, but rather that there is need to search for a more effective source of 
authority. Because the older education imposed the knowledge, methods, and the rules of 
conduct of the mature person upon the young, it does not follow, except upon the basis of 
the extreme Either-Or philosophy, that the knowledge and skill of the mature person has 
no directive value for the experience of the immature. On the contrary, basing education 
upon personal experience may mean more multiplied and more intimate contacts between 
the mature and the immature than ever existed in the traditional school, and consequently 
more, rather than less, guidance by others. The problem, then, is: how these contacts can 
be established without violating the principle of learning through personal experience. 
The solution of this problem requires a well thought-out philosophy of the social factors 
that operate in the constitution of individual experience. 

What is indicated in the foregoing remarks is that the general principles of the new 
education do not of themselves solve any of the problems of the actual or practical 
conduct and management of progressive schools. Rather, they set new problems which 
have to be worked out on the basis of a new philosophy of experience. The problems are 
not even recognized, to say nothing of being solved, when it is assumed that it suffices to 
reject the ideas and practices of the old education and then go to the opposite extreme. 
Yet I am sure that you will appreciate what is meant when I say that many of the newer 
schools tend to make little or nothing of organized subject-matter of study; to proceed as 
if any form of direction and guidance by adults were an invasion of individual freedom, 
and as if the idea that education should be concerned with the present and future meant 
that acquaintance with the past has little or no role to play in education. Without pressing 
these defects to the point of exaggeration, they at least illustrate what is meant by a 
theory and practice of education which proceeds negatively or by reaction against what 
has been current in education rather than by a positive and constructive development of 
purposes, methods, and subject-matter on the foundation of a theory of experience and its 
educational potentialities. 

It is not too much to say that an educational philosophy which professes to be based 
on the idea of freedom may become as dogmatic as ever was the traditional education 
which is reacted against. For any theory and set of practices is dogmatic which is not 
based upon critical examination of its own underlying principles. Let ns say that the new 
education emphasizes the freedom of the learner. Very well. A problem is now set. What 
does freedom mean and what are the conditions under which it is capable of realization? 



Let us say that the kind of eternal imposition which was so common in the traditional 
school limited rather than promoted the intellectual and moral development of the young. 
Again, very well. Recognition of this serious defect sets a problem. Just what is the role 
of the teacher and of books in promoting the educational development of the immature 
Admit that traditional education employed as the subject-matter for study facts and ideas 
so bound up with the past as to give little help in dealing with the issues of the present 
and future. Very well. Now we have the problem of discovering the connection which 
actually exists within experience between the achievements of the past and the issues of 
the present. We have the problem of ascertaining how acquaintance with the past may be 
translated into a potent instrumentality for dealing effectively with the future. We may 
reject knowledge of the past as the end of education and thereby only emphasize its 
importance as a means. When we do that we have a problem that is new in the story of 
education: How shall the young become acquainted with the past in such a way that the 
acquaintance is a potent agent in appreciation of the living present? 


Chapter 2 

The Need of a Theory of Experience 

IN SHORT, the point I am making is that rejection of the philosophy and practice of 
traditional education sets a new type of difficult educational problem for those who 
believe in the new type of education. We shall operate blindly and in confusion until we 
recognize this fact; until we thoroughly appreciate that departure from the old solves no 
problems. What is said in the following pages is, accordingly, intended to indicate some 
of the main problems with which the newer education is confronted and to suggest the 
main lines along which their solution is to be sought. I assume that amid all uncertainties 
there is one permanent frame of reference: namely, the organic connection between 
education and personal experience; or, that the new philosophy of education is committed 
to some kind of empirical and experimental philosophy. But experience and experiment 
are not self-explanatory ideas. Rather, their meaning is part of the problem to be 
explored. To know the meaning of empiricism we need to understand what experience is. 

The belief that ah genuine education comes about through experience does not mean 
that all experiences are genuinely or equally educative. Experience and education cannot 
be directly equated to each other. For some experiences are miseducative. Any 
experience is miseducative that has the effect of arresting or distorting the growth of 
further experience. An experience may be such as to engender callousness; it may 
produce lack of sensitivity and of responsiveness. Then the possibilities of having richer 
experience in the future are restricted. Again, a given experience may increase a person's 
automatic skill in a particular direction and yet tend to land him in a groove or rut; the 
effect again is to narrow the field of further experience. An experience may be 
immediately enjoyable and yet promote the formation of a slack and careless attitude; this 
attitude then operates to modify the quality of subsequent experiences so as to prevent a 
person from getting out of them what they have to give. Again, experiences may be so 
disconnected from one another that, while each is agreeable or even exciting in itself, 


they are not linked cumulatively to one another. Energy is then dissipated and a person 
becomes scatter- brained. Each experience may be lively, vivid, and "interesting," and yet 
their disconnectedness may artificially generate dispersive, disintegrated, centrifugal 
habits. The consequence of formation of such habits is inability to control future 
experiences. They are then taken, either by way of enjoyment or of discontent and revolt, 
just as they come. Linder such circumstances, it is idle to talk of self-control. 

Traditional education offers a plethora of examples of experiences of the kinds just 
mentioned. It is a great mistake to suppose, even tacitly, that the traditional schoolroom 
was not a place in which pupils had experiences. Yet this is tacitly assumed when 
progressive education as a plan of learning by experience is placed in sharp opposition to 
the old. The proper line of attack is that the experiences, which were had, by pupils and 
teachers alike, were largely of a wrong kind. How many students, for example, were 
rendered callous to ideas, and how many lost the impetus to learn because of the Way in 
which learning was experienced by them? How many acquired special skills by means of 
automatic drill so that their power of judgment and capacity to act intelligently in new 
situations was limited? How many came to associate the learning process with ennui and 
boredom? How many found what they did learn so foreign to the situations of life outside 
the school as to give them no power of control over the latter? How many came to 
associate books with dull drudgery, so that they were "conditioned" to all but flashy 
reading matter? 

If I ask these questions, it is not for the sake of whole sale condemnation of the old 
education. It is for quite another purpose. It is to emphasize the fact, first, that young 
people in traditional schools do have experiences; and, secondly, that the trouble is not 
the absence of experiences, but their defective and wrong character— wrong and defective 
from the standpoint of connection with further experience. The positive side of this point 
is even more important in connection with progressive education. It is not enough to 
insist upon the necessity of experience, nor even of activity in experience Every- thing 
depends upon the quality of the experience, which is had. The quality of any experience 
has two aspects. There is an immediate aspect of agreeableness or disagreeableness, and 
there is its influence upon later experiences. The first is obvious and easy to judge. The 
effect of an experience is not borne on its face. It sets a problem to the educator. It is his 
business to arrange for the kind of experiences which, while they do not repel the student, 
but rather engage his activities are, nevertheless, more than immediately enjoyable since 
they promote having desirable future experiences Just as no man lives or dies to himself, 
so no experience lives and dies to itself. Wholly independent of desire or intent every 
experience lives on in further experiences. Hence the central problem of an education 
based upon experience is to select the kind of present experiences that live fruitfully and 
creatively in subsequent experiences. 

Later, I shall discuss in more detail the principle of the continuity of experience or 
what may be called the experiential continuum. Here I wish simply to emphasize the 
importance of this principle for the philosophy of educative experience. A philosophy of 
education, like any theory, has to be stated in words, in symbols. But so far as it is more 
than verbal it is a plan for conducting education. Like any plan, it must be framed with 



reference to what is to be done and how it is to be done. The more definitely and 
sincerely it is held that education is a development within, by, and for experience, the 
more important it is that there shall be clear conceptions of what experience is. Unless 
experience is so conceived that the result is a plan for deciding upon subject-matter, upon 
methods of instruction and discipline, and upon material equipment and social 
organization of the school, it is wholly in the air. It is reduced to a form of words which 
may be emotionally stirring but for which any other set of words might equally well be 
substituted unless they indicate operations to be initiated and executed. Just because 
traditional education was a matter of routine in which the plans and programs were 
handed down from the past, it does not follow that progressive education is a matter of 
planless improvisation. 

The traditional school could get along without any consistently developed philosophy 
of education. About all it required in that line was a set of abstract words like culture, 
discipline, our great cultural heritage, etc., actual guidance being derived not from them 
but from custom and established routines. Just because progressive schools cannot rely 
upon established traditions and institutional habits, they must either proceed more or less 
haphazardly or be directed by ideas which, when they are made articulate and coherent, 
form a philosophy of education. Revolt against the kind of organization characteristic of 
the traditional school constitutes a demand for a kind of organization based upon ideas. I 
think that only slight acquaintance with the history of education is needed to prove that 
educational reformers and innovators alone have felt the need for a philosophy of 
education. Those who adhered to the established system needed merely a few fine- 
sounding words to justify existing practices. The real work was done by habits, which 
were so fixed as to be institutional. The lesson for progressive education is that it requires 
in an urgent degree, a degree more pressing than was incumbent upon former innovators, 
a philosophy of education based upon a philosophy of experience. 

I remarked incidentally that the philosophy in question is, to paraphrase the saying of 
Lincoln about democracy, one of education of, by, and for experience. No one of these 
words, of, by, or for, names anything which is self- evident. Each of them is a challenge 
to discover and put into operation a principle of order and organization, which follows 
from understanding what educative experience, signifies. 

It is, accordingly, a much more difficult task to work out the kinds of materials, of 
methods, and of social relationships that are appropriate to the new education than is the 
case with traditional education. I think many of the difficulties experienced in the conduct 
of progressive schools and many of the criticisms leveled against them arise from this 
source. The difficulties are aggravated and the criticisms are increased when it is 
supposed that the new education is somehow easier than the old. This belief is, I imagine, 
more or less current. Perhaps it illustrates again the Either-Or philosophy, springing from 
the idea that about all which is required is not to do what is done in traditional schools. 

I admit gladly that the new education is simpler in principle than the old. It is in 
harmony with principles of growth, while there is very much which is artificial in the old 
selection and arrangement of subjects and methods, and artificiality always leads to 



unnecessary complexity. But the easy and the simple are not identical. To discover what 
is really simple and to act upon the discovery is an exceedingly difficult task. After the 
artificial and complex is once institutionally established and ingrained in custom and 
routine, it is easier to walk in the paths that have been beaten than it is, after taking a new 
point of view, to work out what is practically involved in the new point of view. The old 
Ptolemaic astronomical system was more complicated with its cycles and epicycles than 
the Copernican system. But until organization of actual astronomical phenomena on the 
ground of the latter principle had been effected the easiest course was to follow the line 
of least resistance provided by the old intellectual habit. So we come back to the idea that 
a coherent theory of experience, affording positive direction to selection and organization 
of appropriate educational methods and materials, is required by the attempt to give new 
direction to the work of the schools. The process is a slow and arduous one. It is a matter 
of growth and there are many obstacles, which tend to obstruct growth and to deflect it 
into wrong lines. 

I shall have something to say later about organization. All that is needed, perhaps, at 
this point is to say that we must escape from the tendency to think of organization in 
terms of the land of organization, whether of content (or subject-matter), or of methods 
and social relations, that mark traditional education. I think that a good deal of the current 
opposition to the idea of organization is due to the fact that it is so hard to get away from 
the picture of the studies of the old school. The moment "organization" is mentioned 
imagination goes almost automatically to the kind of organization that is familiar, and in 
revolting against that we are led to shrink from the very idea of any organization. On the 
other hand, educational reactionaries, who are now gathering force, use the absence of 
adequate intellectual and moral organization in the newer type of school as proof not only 
of the need of organization, but to identify any and every kind of organization with that 
instituted before the rise of experimental science. Failure to develop a conception of 
organization upon the empirical and experimental basis gives reactionaries a too easy 
victory. But the fact that the empirical sciences now offer the best type of intellectual 
organization which can be found in any field shows that there is no reason why we, who 
call ourselves empiricists, should be "pushovers" in the matter of order and organization. 


Chapter 3 

Criteria of Experience 

IF THERE IS any truth in what has been said about the need of forming a theory of 
experience in order that education may be intelligently conducted upon the basis of 
experience, it is clear that the next thing in order in this discussion is to present the 
principles that are most significant in framing this theory. I shall not; therefore, apologize 
for engaging in a certain amount of philosophical analysis, which otherwise might be out 
of place. I may, however, reassure you to some degree by saying that this analysis is not 
an end in itself but is engaged in for the sake of obtaining criteria to be applied later in 
discussion of a number of concrete and, to most persons, more interesting issues. 


I have already mentioned what I called the category of continuity, or the experiential 
continuum. This principle is involved, as I pointed out, in every attempt to discriminate 
between experiences that are worth while educationally and those that are not. It may 
seem superfluous to argue that this discrimination is necessary not only in criticizing the 
traditional type of education but also in initiating and conducting a different type. 
Nevertheless, it is advisable to pursue for a little while the idea that it is necessary. One 
may safely assume, I suppose, that one thing which has recommended the progressive 
movement is that it seems more in accord with the democratic ideal to which our people 
is committed than do the procedures of the traditional school, since the latter have so 
much of the autocratic about them. Another thing which has contributed to its favorable 
reception is that its methods are humane in comparison with the harshness so often 
attending the policies of the traditional school. 

The question I would raise concerns why we prefer democratic and humane 
arrangements to those, which are autocratic and harsh. And by "why," I mean the reason 
for preferring them, not just the causes which lead us to the preference. One cause may be 
that we have been taught not only in the schools but by the press, the pulpit, the platform, 
and our laws and law-making bodies that democracy is the best of all social institutions. 
We may have so assimilated this idea from our surroundings that it has become an 
habitual part of our mental and moral make-up. But similar causes have led other persons 
in different surroundings to widely varying conclusions— to prefer fascism, for example. 
The cause for our preference is not the same thing as the reason why we should prefer it. 

It is not my purpose here to go in detail into the reason. But I would ask a single 
question: Can we find any reason that does not ultimately come down to the belief that 
democratic social arrangements promote a better quality of human experience, one which 
is more widely accessible and enjoyed, than do non-democratic and anti-democratic 
forms of social life? Does not the principle of regard for individual freedom and for 
decency and kindliness of human relations come back in the end to the conviction that 
these things are tributary to a higher quality of experience on the part of a greater number 
than are methods of repression and coercion or force? Is it not the reason for our 
preference that we believe that mutual consultation and convictions reached through 
persuasion, make possible a better quality of experience than can otherwise be provided 
on any wide scale? 

If the answer to these questions is in the affirmative (and personally I do not see how 
we can justify our preference for democracy and humanity on any other ground), the 
ultimate reason for hospitality to progressive education, because of its reliance upon and 
use of humane methods and its kinship to democracy, goes back to the fact that 
discrimination is made between the inherent values of different experiences. So I come 
back to the principle of continuity of experience as a criterion of discrimination. 

At bottom, this principle rests upon the fact of habit, when habit is interpreted 
biologically. The basic characteristic of habit is that every experience enacted and 
undergone modifies the one who acts and undergoes, while this modification affects, 
whether we wish it or not, the quality of subsequent experiences. For it is a somewhat 



different person who enters into them. The principle of habit so understood obviously 
goes deeper than the ordinary conception of a habit as a more or less fixed way of doing 
things, although it includes the latter as one of its special cases. It covers the formation of 
attitudes, attitudes that are emotional and intellectual; it covers our basic sensitivities and 
ways of meeting and responding to all the conditions which we meet in living. From this 
point of view, the principle of continuity of experience means that every experience both 
takes up something from those which have gone before and modifies in some way the 
quality of those which come after. As the poet states it, 

... all experience is an arch wherethro' 

Gleams that untraveled world, whose margin fades 

Forever and forever when I move. 

So far, however, we have no ground for discrimination among experiences. For the 
principle is of universal application. There is some kind of continuity in every case. It is 
when we note the different forms in which continuity of experience operates that we get 
the basis of discriminating among experiences. I may illustrate what is meant by an 
objection, which has been brought against an idea which I once put forth— namely, that 
the educative process can be identified with growth when that is understood in terms of 
the active participle, growing. 

Growth, or growing as developing, not only physically but intellectually and morally, 
is one exemplification of the principle of continuity. The objection made is that growth 
might take many different directions: a man, for example, who starts out on a career of 
burglary may grow in that direction, and by practice may grow into a highly expert 
burglar. Hence it is argued that "growth" is not enough; we must also specify the 
direction in which growth takes place, the end towards which it tends. Before, however, 
we decide that the objection is conclusive we must analyze the case a little further. 

That a man may grow in efficiency as a burglar, as a gangster, or as a corrupt 
politician, cannot be doubted. But from the standpoint of growth as education and 
education as growth the question is whether growth in this direction promotes or retards 
growth in general. Does this form, of growth create conditions for further growth, or does 
it set up conditions that shut oh the person who has grown in this particular direction 
from the occasions, stimuli, and opportunities for continuing growth in new directions? 
What is the effect of growth in a special direction upon the attitudes and habits which 
alone open up avenues for development in other lines? I shall leave you to answer these 
questions, saying simply that when and only when development in a particular line 
conduces to continuing growth does it answer to the criterion of education as growing. 
For the conception is one that must find universal and not specialized limited application. 

I return now to the question of continuity as a criterion by which to discriminate 
between experiences which are educative and those which are mis-educative. As we have 
seen, there is some kind of continuity in any case since every experience affects for better 



or worse the attitudes which help decide the quality of further experiences, by setting up 
certain preference and aversion, and making it easier or harder to act for this or that end. 
Moreover, every experience influences in some degree the objective conditions under 
which further experiences are had. For example, a child who learns to speak has a new 
facility and new desire. But he has also widened the external conditions of subsequent 
learning. When he learns to read, he similarly opens up a new environment. If a person 
decides to become a teacher, lawyer, physician, or stock-broker, when he executes his 
intention he thereby necessarily determines to some extent the environment in which he 
will act in the future. He has rendered himself more sensitive and responsive to certain 
conditions, and relatively immune to those things about him that would have been stimuli 
if he had made another choice. 

But, while the principle of continuity applies in some way in every case, the quality of 
the present experience influences the way in which the principle applies. We speak of 
spoiling a child and of the spoilt child. The effect of over-indulging a child is a 
continuing one. It sets up an attitude, which operates as an automatic demand that persons 
and objects cater to his desires and caprices in the future. It makes him seek the kind of 
situation that will enable him to do what he feels like doing at the time. It renders him 
averse to and comparatively incompetent in situations, which require effort and 
perseverance in overcoming obstacles. There is no paradox in the fact that the principle 
of the continuity of experience may operate so as to leave a person arrested on a low 
plane of development, in a way, which limits later capacity for growth. 

On the other hand, if an experience arouses curiosity, strengthens initiative, and sets 
up desires and purposes that are sufficiently intense to carry a person over dead places in 
the future, continuity works in a very different way. Every experience is a moving force. 
Its value can be judged only on the ground of what it moves toward and into. The greater 
maturity of experience which should belong to the adult as educator puts him in a 
position to evaluate each experience of the young in a way in which the one having the 
less mature experience cannot do. It is then the business of the educator to see in what 
direction an experience is heading. There is no point in his being more mature if, instead 
of using his greater insight to help organize the conditions of the experience of the 
immature, he throws away his insight. Failure to take the moving force of an experience 
into account so as to judge and direct it on the ground of what it is moving into means 
disloyalty to the principle of experience itself. The disloyalty operates in two directions. 
The educator is false to the understanding that he should have obtained from his own past 
experience. He is also unfaithful to the fact that all human experience is ultimately social: 
that it involves contact and communication. The mature person, to put it in moral terms, 
has no right to withhold from the young on given occasions whatever capacity for 
sympathetic understanding his own experience has given him. 

No sooner, however, are such things said than there is a tendency to read to the other 
extreme and take what has been said as a plea for some sort of disguised imposition from 
outside. It is worth while, accordingly, to say something about the way in which the adult 
can exercise the wisdom his own wider experience gives him without imposing a merely 
external control. On one side, it is his business to be on the alert to see what attitudes and 



habitual tendencies are being created. In this direction he must, if he is an educator, be 
able to judge what attitudes are actually conducive to continued growth and what are 
detrimental. He must, in addition, have that sympathetic understanding ~ individuals as 
individuals which gives him an idea of what is actually going on in the minds of those 
who are learning. It is, among other things, the need for these abilities on the part of the 
parent and teacher which makes a system of education based upon living experience, 
difficult affair to conduct successfully than it is to follow the patterns of traditional 
education. 

But there is another aspect of the matter. Experience does not go on simply inside a 
person. It does go on there, for it influences the formation of attitudes of desire and 
purpose. But this is not the whole of the story. Every genuine experience has an active 
side which changes in some degree the objective conditions under which experiences are 
had. The difference between civilization and savagery, to take an example on a large 
scale, is found in the degree in which previous experiences have changed the objective 
conditions under which subsequent experiences take place. The existence of roads, of 
means of rapid movement and transportation, tools, implements, furniture, electric light 
and power, are illustrations. Destroy the external conditions of present civilized 
experience, and for a time our experience would relapse into that of barbaric peoples. 

In a word, we live from birth to death in a world of persons and things which in large 
measure is what it is because of what has been done and transmitted from previous 
human activities. When this fact is ignored, experience is treated as if it were something 
which goes on exclusively inside an individual's body and mind. It ought not to be 
necessary to say that experience does not occur in a vacuum. There are sources outside an 
individual which give rise to experience. It is constantly fed from these springs. No one 
would question that a child in a slum tenement has a different experience from that of a 
child in a cultured home; that the country lad has a different kind of experience from the 
city boy, or a boy on the seashore one different from the lad who is brought up on inland 
prairies. Ordinarily we take such facts for granted as too commonplace to record. But 
when their educational import is recognized, they indicate the second way in which the 
educator can direct the experience of the young without engaging in imposition. A 
primary responsibility of educators is that they not only be aware of the general principle 
of the shaping of actual experience by environing conditions, but that they also recognize 
in the concrete what surroundings are conducive to having experiences that lead to 
growth. Above all, they should know how to utilize the surroundings, physical and social, 
that exist so as to extract from them all that they have to contribute to building up 
experiences that are worth while. 

Traditional education did not have to face this problem; it could systematically dodge 
this responsibility. The school environment of desks, blackboards, a small schoolyard, 
was supposed to suffice. There was no demand that the teacher should become intimately 
acquainted with the conditions of the local community, physical, historical, economic, 
occupational etc., in order to utilize them as educational resources. A system of education 
based upon the necessary connection of education with experience must, on the contrary, 
if faithful to its principle, take these things constantly into account. This tax upon the 



educator is another reason why progressive education is more difficult to carry on than 
was ever the traditional system. 

It is possible to frame schemes of education that pretty systematically subordinate 
objective conditions to those which reside in the individuals being educated. This 
happens whenever the place and function of the teacher, of books, of apparatus and 
equipment, of everything which represents the products of the more mature experience of 
elders, is systematically subordinated to the immediate inclinations and feelings of the 
young. Every theory which assumes that importance can be attached to these objective 
factors only at the expense of imposing external control and of limiting the freedom of 
individuals rests finally upon the notion that experience is truly experience only when 
objective conditions are subordinated to what goes on within the individuals having the 
experience. 

I do not mean that it is supposed that objective conditions can be shut out. It is 
recognized that they must enter in: so much concession is made to the inescapable fact 
that we live in a world of things and persons. But I think that observation of what goes on 
in some families and some schools would disclose that some parents and some teachers 
are acting upon the idea of subordinating objective conditions to internal ones. In that 
case, it is assumed not only that the latter are primary, which in one sense they are, but 
that just as they temporarily exist they fix the whole educational process. 

Let me illustrate from the case of an infant. The needs of a baby for food, rest, and 
activity are certainly primary and decisive in one respect. Nourishment must be provided; 
provision must be made for comfortable sleep, and so on. But these facts do not mean 
that a parent shall feed the baby at any time when the baby is cross or irritable, that there 
shall not be a program of regular hours of feeding and sleeping, etc. The wise mother 
takes account of the needs of the infant but not in a way, which dispenses with her own 
responsibility for regulating the objective conditions under which the needs are satisfied. 
And if she is a wise mother in this respect, she draws upon past experiences of experts as 
well as her own for the light that these shed upon what experiences are in general most 
conducive to the normal development of infants. Instead of these conditions being 
subordinated to the immediate internal condition of the baby, they are definitely ordered 
so that a particular kind of interaction with these immediate internal states may be 
brought about. 

The word "interaction," which has just been used, expresses the second chief principle 
for interpreting an experience in its educational function and force. It assigns equal rights 
to both factors in experience-objective and internal conditions. Any normal experience is 
an interplay of these two sets of conditions. Taken together, or in their interaction, they 
form what we call a situation. The trouble with traditional education was not that it 
emphasized the external conditions that enter into the control of the experiences but that 
it paid so little attention to the internal factors which also decide what kind of experience 
is had. It violated the principle of interaction from one side. But this violation is no 
reason why the new education should violate the principle from the other side-except 



upon the basis of the extreme Either-Or educational philosophy which has been 
mentioned. 

The illustration drawn from the need for regulation of the objective conditions of a 
baby's development indicates, first, that the parent has responsibility for arranging the 
conditions under which an infant's experience of food, sleep, etc., occurs, and, secondly, 
that the responsibility is fulfilled by utilizing the funded experience of the past, as this is 
represented, say, by the advice of competent physicians and others who have made a 
special study of normal physical growth. Does it limit the freedom of the mother when 
she uses the body of knowledge thus provided to regulate the objective conditions of 
nourishment and sleep? Or does the enlargement of her intelligence in fulfilling her 
parental function widen her freedom? Doubtless if a fetish were made of the advice and 
directions so that they came to be inflexible dictates to be followed under every possible 
condition, then restriction of freedom of both parent and child would occur. But this 
restriction would also be a limitation of the intelligence that is exercised in personal 
judgment. 

In what respect does regulation of objective conditions limit the freedom of the baby? 
Some limitation is certainly placed upon its immediate movements and inclinations when 
it is put in its crib, at a time when it wants to continue playing, or does not get food at the 
moment it would like it, or when it isn't picked up and dandled when it cries for attention. 
Restriction also occurs when mother or nurse snatches a child away from an open fire 
into which it is about to fall. I shall have more to say later about freedom. Here it is 
enough to ask whether freedom is to be thought of and adjudged on the basis of relatively 
momentary incidents or whether its meaning is found in the continuity of developing 
experience. 

The statement that individuals live in a world means, in the concrete, that they live in 
a series of situations. And when it is said that they live in these situations, the meaning of 
the word "in" is different from its meaning when it is said that pennies are "in" a pocket 
or paint is "in" a can. It means, once more, that interaction is going on between an 
individual and objects and other persons. The conceptions of situation and of interaction 
are inseparable from each other. An experience is always what it is because of a 
transaction taking place between an individual and what, at the time, constitutes his 
environment, whether the latter consists of persons with whom he is talking about some 
topic or event, the subject talked about being also a part of the situation; or the toys with 
which he is playing; the book he is reading (in which his environing conditions at the 
time may be England or ancient Greece or an imaginary region); or the materials of an 
experiment he is performing. The environment, in other words, is whatever conditions 
interact with personal needs, desires, purposes, and capacities to create the experience 
which is had. Even when a person builds a castle in the air he is interacting with the 
objects which he constructs in fancy. 

The two principles of continuity and interaction are not separate from each other. 
They intercept and unite. They are, so to speak, the longitudinal and lateral aspects of 
experience. Different situations succeed one another. But because of the principle of 



continuity something is carried over from the earlier to the later ones. As an individual 
passes from one situation to another, his world, his environment, expands or contracts. 
He does not find himself living in another world but in a different part or aspect of one 
and the same world. What he has learned in the way of knowledge and skill in one 
situation becomes an instrument of understanding and dealing effectively with the 
situations which follow. The process goes on as long as life and learning continue. 
Otherwise the course of experience is disorderly, since the individual factor that enters 
into making an experience is split. A divided world, a world whose parts and aspects do 
not hang together, is at once a sign and a cause of a divided personality. When the 
splitting-up reaches a certain point we call the person insane. A fully integrated 
personality, on the other hand, exists only when successive experiences are integrated 
with one another. It can be built up only as a world of related objects is constructed. 

Continuity and interaction in their active union with each other provide the measure of 
the educative significance and value of an experience. The immediate and direct concern 
of an educator is then with the situations in which interaction takes place. The individual, 
who enters as a factor into it, is what he is at a given time. It is the other factor, that of 
objective conditions, which lies to some extent within the possibility of regulation by the 
educator. As has already been noted, the phrase "objective conditions" covers a wide 
range. It includes what is done by the educator and the way in which it is done, not only 
words spoken but the tone of voice in which they are spoken. It includes equipment, 
books, apparatus, toys, games played. It includes the materials with which an individual 
interacts, and, most important of all, the total social set-up of the situations in which a 
person is engaged. 

When it is said that the objective conditions are those which are within the power of 
the educator to regulate, it is meant, of course, that his ability to influence directly the 
experience of others and thereby the education they obtain places upon him the duty of 
determining that environment which will interact with the existing capacities and needs 
of those taught to create a worth-while experience. The trouble with traditional education 
was not that educators took upon themselves the responsibility for providing an 
environment. The trouble was that they did not consider the other factor in creating an 
experience; namely, the powers and purposes of those taught. It was assumed that a 
certain set of conditions was intrinsically desirable, apart from its ability to evoke a 
certain quality of response in individuals. This lack of mutual adaptation made the 
process of teaching and learning accidental. Those to whom the provided conditions were 
suitable managed to learn. Others got on as best they could. Responsibility for selecting 
objective conditions carries with it, then, the responsibility for understanding the needs 
and capacities of the individuals who are learning at a given time. It is not enough that 
certain materials and methods have proved effective with other individuals at other times. 
There must be a reason for thinking that they will function in generating an experience 
that has educative quality with particular individuals at a particular time. 

It is no reflection upon the nutritive quality of beefsteak that it is not fed to infants. It 
is not an invidious reflection upon trigonometry that we do not teach it in the first or fifth 
grade of school. It is not the subject per se that is educative or that is conducive to 



growth. There is no subject that is in and of itself, or without regard to the stage of 
growth attained by the learner, such that inherent educational value can be attributed to it. 
Failure to take into account adaptation to the needs and capacities of individuals was the 
source of the idea that certain subjects and certain methods are intrinsically cultural or 
intrinsically good for mental discipline. There is no such thing as educational value in the 
abstract. The notion that some subjects and methods and that acquaintance with certain 
facts and truths possess educational value in and of themselves is the reason why 
traditional education reduced the material of education so largely to a diet of predigested 
materials. According to this notion, it was enough to regulate the quantity and difficulty 
of the material provided, in a scheme of quantitative grading, from month to month and 
from year to year. Otherwise a pupil was expected to take it in doses that were prescribed 
from without. If the pupil left it in- stead of taking it, if he engaged in physical truancy, or 
in the mental truancy of mind-wandering and finally built up an emotional revulsion 
against the subject, he was held to be at fault. No question was raised as to whether the 
trouble might not lie in the subject-matter or in the way in which it was offered. The 
principle of interaction makes it clear that failure of adaptation of material to needs and 
capacities of individuals may cause an experience to be non-educative quite as much as 
failure of an individual to adapt himself to the material. 

The principle of continuity in its educational application means, nevertheless, that the 
future has to be taken into account at every stage of the educational process. This idea is 
easily misunderstood and is badly distorted in traditional education. Its assumption is, 
that by acquiring certain skills and by learning certain subjects which would be needed 
later (perhaps in college or perhaps in adult life) pupils are as a matter of course made 
ready for the needs and circumstances of the future. Now "preparation" is a treacherous 
idea. In a certain sense every experience should do something to prepare a person for 
later experiences of a deeper and more expansive quality. That is the very meaning of 
growth, continuity, reconstruction of experience. But it is a mistake to suppose that the 
mere acquisition of a certain amount of arithmetic, geography, history, etc., which is 
taught and studied because it may be useful at some time in the future, has this effect, and 
it is a mistake to suppose that acquisition of skills in reading and figuring will 
automatically constitute preparation for their right and effective use under conditions 
very unlike those in which they were acquired. 

Almost everyone has had occasion to look back upon his school days and wonder 
what has become of the knowledge he was supposed to have amassed during his years of 
schooling, and why it is that the technical skills he acquired have to be learned over again 
in changed form in order to stand him in good stead. Indeed, he is lucky who does not 
find that in order to make progress, in order to go ahead intellectually, he does not have 
to unlearn much of what he learned in school. These questions cannot be disposed of by 
saying that the subjects were not actually learned for they were learned at least 
sufficiently to enable a pupil to pass examinations in them. One trouble is that the 
subject-matter in question was learned in isolation; it was put, as it were, in a water-tight 
compartment. When the question is asked, then, what has become of it, where has it gone 
to, the right answer is that it is still there in the special compartment in which it was 
originally stowed away. If exactly the same conditions recurred as those under which it 



was acquired, it would also recur and be available. But it was segregated when it was 
acquired and hence is so disconnected from the rest of experience that it is not available 
under the actual conditions of life. It is contrary to the laws of experience that learning of 
this kind, no matter how thoroughly engrained at the time, should give genuine 
preparation. 

Nor does failure in preparation end at this point. Perhaps the greatest of all 
pedagogical fallacies is the notion that it person learns only the particular thing he is 
studying at the time. Collateral learning in the way of formation of enduring attitudes, of 
likes and dislikes, may be and often is much more important than the spelling lesson or 
lesson in geography or history that is learned. For these attitudes are fundamentally what 
count in the future. The most important attitude that can be formed is that of desire to go 
on learning. If impetus in this direction is weakened instead of being intensified, 
something much more than mere lack of preparation takes place. The pupil is actually 
robbed of native capacities which otherwise would enable him to cope with the 
circumstances that he meets in the course of his life. We often see persons who have had 
little schooling and in whose case the absence of set schooling proves to be a positive 
asset. They have at least retained their native common sense and power of judgment, and 
its exercise in the actual conditions of living has given them the precious gift of ability to 
learn from the experiences they have. What avail is it to win pre scribed amounts of 
information about geography and history, to win ability to read and write, if in the 
process the individual loses his own soul: loses his appreciation of things worth while, of 
the values to which these things are relative; if he loses desire to apply what he has 
learned and, above all, loses the ability to extract meaning from his future experiences as 
they occur? 

What, then, is the true meaning of preparation in the educational scheme? In the first 
place, it means that a person, young or old, gets out of his present experience all that 
there is in it for him at the time in which he has it. When preparation is made the 
controlling end, then the potentialities of the present are sacrificed to a suppositious 
future. When this happens, the actual preparation for the future is missed or distorted. 
The ideal of using the present simply to get ready for the future contradicts itself. It 
omits, and even shuts out, the very conditions by which a person can be prepared for his 
future. We always live at the time we live and not at some other time, and only by 
extracting at each present time the full meaning of each present experience are we 
prepared for doing the same thing in the future. This is the only preparation which in the 
long run amounts to anything. 

All this means that attentive care must be devoted to the conditions which give each 
present experience a worthwhile meaning. Instead of inferring that it doesn't make much 
difference what the present experience is as long as it is enjoyed, the conclusion is the 
exact opposite. Here is another matter where it is easy to react from one extreme to the 
other. Because traditional schools tended to sacrifice the present to a remote and more or 
less unknown future, therefore it comes to be believed that the educator has little 
responsibility for the kind of present experiences the young undergo. But the relation of 
the present and the future is not an Either-Or affair. The present affects the future 



anyway. The persons who should have some idea of the connection between the two are 
those who have achieved maturity. Accordingly, upon them devolves the responsibility 
for instituting the conditions for the land of present experience which has a favorable 
effect upon the future. Education as growth or maturity should be an ever-present 
process. 


Chapter 4 


Social Control 


I HAVE Said that educational plans and projects, seeing education in terms of life 
experience, are thereby committed to framing and adopting an intelligent theory or, if you 
please, philosophy of experience. Otherwise they are at the mercy of every intellectual 
breeze that happens to blow. I have tried to illustrate the need for such a theory by calling 
attention to two principles, which are fundamental in the constitution of experience: the 
principles of interaction and of continuity. If, then, I am asked why I have spent so much 
time on expounding a rather abstract philosophy, it is because practical attempts to 
develop schools based upon the idea that education is found in life-experience are bound 
to exhibit inconsistencies and confusions unless they are guided by some conception of 
what experience is, and what marks oh educative experience from non-educative and 
mis-educative experience. I now come to a group of actual educational questions the 
discussion of which will, I hope, provide topics and material that are more concrete than 
the discussion up to this point. 

The two principles of continuity and interaction as criteria of the value of experience 
are so intimately connected that it is not easy to tell just what special educational problem 
to take up first. Even the convenient division into problems of subject-matter or studies 
and of methods of teaching and learning is likely to fail us in selection and organization 
of topics to discuss. Consequently, the beginning and sequence of topics is somewhat 
arbitrary. I shall commence, however, with the old question of individual freedom and 
social control and pass on to the questions that grow naturally out of it. 

It is often well in considering educational problems to get a start by temporarily 
ignoring the school and thinking of other human situations. I take it that no one would 
deny that the ordinary good citizen is as a matter of fact subject to a great deal of social 
control and that a considerable part of this control is not felt to involve restriction of 
personal freedom. Even the theoretical anarchist, whose philosophy commits him to the 
idea that state or government control is an unmitigated evil, believes that with abolition 
of the political state other forms of social control would operate: indeed, his opposition to 
govern- mental regulation springs from his belief that other and to him more normal 
modes of control would operate with abolition of the state. 

Without taking up this extreme position, let us note some examples of social control 
that operate in everyday life, and then look for the principle underlying them. Let us 
begin with the young people themselves. Children at recess or after school play games, 


from tag and one-old- cat to baseball and football. The games involve rules, and these 
rules order their conduct. The games do not go on haphazardly or by a succession of 
improvisations. Without rules there is no game. If disputes arise there is an umpire to 
appeal to, or discussion and a kind of arbitration are means to a decision; otherwise the 
game is broken up and comes to an end. 

There are certain fairly obvious controlling features of such situations to which I want 
to call attention. The first is that the rules are a part of the game. They ate not outside of 
it. No rules, then no game; different rules, then a different game. As long as the game 
goes on with a reasonable smoothness, the players do not feel that they are submitting to 
external imposition but that they are playing the game. In the second place at times feel 
that a decision isn't fair and be may even get angry. But he is not objecting to a rule but to 
what he claims is a violation of it, to some one-sided and unfair action. In the third place, 
the rules, and hence the conduct of the game, are fairly standardized. There are 
recognized ways of counting out, of selection of sides, as well as for positions to be 
taken, movements to be made, etc. These rules have the sanction of tradition and 
precedent. Those playing the game have seen, perhaps, professional matches and they 
want to emulate their elders. An element that is conventional is pretty strong. Usually, a 
group of youngsters change the rules by which they play only when the adult group to 
which they look for models have themselves made a change in the rules, while the change 
made by the elders is at least supposed to conduce to making the game more skillful or 
more interesting to spectators. 

Now, the general conclusion I would draw is that control of individual actions is 
effected by the whole situation in which individuals are involved, in which they share and 
of which they are co-operative or interacting parts. For even in a competitive game there 
is a certain kind of participation, of sharing in a common experience. Stated the other 
way mound, those who take part do not feel that they are bossed by an individual person 
or are being subjected to the will of some outside superior person. When violent disputes 
do arise, it is usually on the alleged ground that the umpire or a person on the other side is 
being unfair; in other words, that in such cases some individual is trying to impose his 
individual will on someone else. 

It may seem to be putting too heavy a load upon a single case to argue that this 
instance illustrates the general principle of social control of individuals without the 
violation of freedom. But if the matter were followed out through a number of cases, I 
think the conclusion that this particular instance does illustrate a general principle would 
be justified. Games are generally competitive. If we took instances of co-operative 
activities in which all members of a group take part, as for example in well-ordered 
family life in which there is mutual confidence, the point would be even clearer. In all 
such cases it is not the will or desire of any one person which establishes order but the 
moving spirit of the whole group. The control is social, but individuals are parts of a 
community, not outside of it. 

I do not mean by this that there are no occasions upon which the authority of, say, the 
parent does not have to intervene and exercise fairly direct control. But I do say that, in 



the first place, the number of these occasions is slight in comparison with the number of 
those in which the control is exercised by situations in which all take part. And what is 
even more important, the authority in question when exercised in a well-regulated 
household or other community group is not a manifestation of merely personal will; the 
parent or teacher exercises it as the representative and agent of the interests of the group 
as a whole. With respect to the first point, in a well ordered school the main reliance for 
control of this and that individual is upon the activities carried on and upon the situations 
in which these activities are maintained. The teacher reduces to a minimum the occasions 
in which he or she has to exercise authority in a personal way. When it is necessary, in 
the second place, to speak and act firmly, it is done in behalf of the interest of the group, 
not as an exhibition of personal power. This makes the difference between action, which 
is arbitrary, and that which is just and fair. 

Moreover, it is not necessary that the difference should be formulated in words, by 
either teacher or the young, in order to be felt in experience. The number of children who 
do not feel the difference (even if they cannot articulate it and reduce it to an intellectual 
principle) between action that is motivated by personal power and desire to dictate and 
action that is fair, because in the interest of all, is small. I should even be willing to say 
that upon the whole children are more sensitive to the signs and symptoms of this 
difference than are adults. Children learn the difference when playing with one another. 
They are willing, often too willing if anything, to take suggestions from one child and let 
him be a leader if his conduct adds to the experienced value of what they are doing, while 
they resent the attempt at dictation. Then they often withdraw and when asked why, say 
that it is because so-and-so "is too bossy." 

I do not wish to refer to the traditional school in ways which set up a caricature in lieu 
of a picture. But I think it is fair to say that one reason the personal commands of the 
teacher so often played an undue role and a season why the order which existed was so 
much a matter of sheer obedience to the will of an adult was because the situation almost 
forced it upon the teacher. The school was not a group or community held together by 
participation in common activities. Consequently, the normal, proper conditions of 
control were lacking. Their absence was made up for, and to a considerable extent had to 
be made up for, by the direct intervention of the teacher, who, as the saying went, "kept 
order." He kept it because order was in the teacher's keeping, instead of residing in the 
shared work being done. 

The conclusion is that in what are called the new schools, the primary source of social 
control resides in the very nature of the work done as a social enterprise in which all 
individuals have an opportunity to contribute and to which all feel a responsibility. Most 
children are naturally "sociable." Isolation is even more irksome to them than to adults. A 
genuine community life has its ground in this natural sociability. But community life does 
not organize itself in an enduring way purely spontaneously. It requires thought and 
planning ahead. The educator is responsible for a knowledge of individuals and for a 
knowledge of subject-matter that will enable activity ties to be selected which lend 
themselves to social organization, an organization in which all individuals have an 



opportunity to contribute something, and in which the activities in which all participate 
are the chief carrier of control. 

I am not romantic enough about the young to suppose that every pupil will respond or 
that any child of normally strong impulses will respond on every occasion. There are 
likely to be some who, when they come to school, are already victims of injurious 
conditions outside of the school and who have become so passive and unduly docile that 
they fail to contribute. There will be others who, because of previous experience, are 
bumptious and unruly and perhaps downright rebellious. But it is certain that the general 
principle of social control cannot be predicated upon such cases. It is also true that no 
general rule can be laid down for dealing with such cases. The teacher has to deal with 
them individually. They fall into general classes, but no two are exactly alike. The 
educator has to discover as best he or she can the causes for the recalcitrant attitudes. He 
or she cannot, if the educational process is to go on, make it a question of pitting one will 
against another in order to see which is strongest, nor yet allow the unruly and non- 
participating pupils to stand permanently in the way of the educative activities of others. 
Exclusion perhaps is the only available measure at a given juncture, but it is no solution. 
For it may strengthen the very causes which have brought about the undesirable anti- 
social attitude, such as desire for attention or to show off. 

Exceptions rarely prove a rule or give a clew to what the rule should be. I would not, 
therefore, attach too much importance to these exceptional cases, although it is true at 
present that progressive schools are likely often to have more than their fair share of these 
cases, since parents may send children to such schools as a last resort. I do not think 
weakness in control when it is found in progressive schools arises in any event from these 
exceptional cases. It is much more likely to arise from failure to arrange in advance for 
the kind of work (by which I mean all kinds of activities engaged in) which will create 
situations that of themselves tend to exercise control over what this, that, and the other 
pupil does and how he does it. This failure most often goes back to lack of sufficiently 
thoughtful planning in advance. The causes for such lack are varied. The one, which is 
peculiarly important to mention in this connection, is the idea that such advance planning 
is unnecessary and even that it is inherently hostile to the legitimate freedom of those 
being instructed. 

Now, of course, it is quite possible to have preparatory planning by the teacher done in 
such a rigid and intellectually inflexible fashion that it does result in adult imposition, 
which is none the less external because executed with tact and the semblance of respect 
for individual freedom. But this kind of planning does not follow inherently from the 
principle involved. I do not know what the greater maturity of the teacher and the 
teacher's greater knowledge of the world, of subject-matters and of individuals, is for 
unless the teacher can arrange conditions that are conducive to community activity and to 
organization which exercises control over individual impulses by the mere fact that all 
are engaged in communal projects. Because the kind of advance planning heretofore 
engaged in has been so routine as to leave little room for the free play of individual 
thinking or for contributions due to distinctive individual experience, it does not follow 
that all planning must be rejected. On the contrary, there is incumbent upon the educator 



the duty of instituting: a much more intelligent, and consequently, more difficult, kind of 
planning. He must survey the capacities and needs of the particular set of individuals with 
whom he is dealing and must at the same time arrange the conditions which provide the 
subject-matter or content for experiences that satisfy these needs and develop these 
capacities. The planning must be flexible enough to permit free play for individuality of 
experience and yet firm enough to give direction towards continuous development of 
power. 

The present occasion is a suitable one to say something about the province and office 
of the teacher. The principle that development of experience comes about through 
interaction means that education is essentially a social process. This quality is realized in 
the degree in which individuals form a community group. It is absurd to exclude the 
teacher from membership in the group. As the most mature member of the group he has a 
peculiar responsibility for the conduct of the interactions and inter- communications 
which are the very life of the group as a community. That children are individuals whose 
freedom should be respected while the more mature person should have no freedom as an 
individual is an idea too absurd to require refutation. The tendency to exclude the teacher 
from a positive and leading share in the direction of the activities of the community of 
which he is a member is another instance of reaction from one extreme to another. When 
pupils were a class rather than a social group, the teacher necessarily acted largely from 
the outside, not as a director of processes of exchange in which all had a share. When 
education is based upon experience and educative experience is seen to be a social 
process, the situation changes radically. The teacher loses the position of external boss or 
dictator but takes on that of leader of group activities. 

In discussing the conduct of games as an example of normal social control, reference 
was made to the presence of a standardized conventional factor. The counterpart of this 
factor in school life is found in the question of manners, especially of good manners in 
the manifestations of politeness and courtesy. The more we know about customs in 
different parts of the world at different times in the history of mankind, the more we learn 
how much manners differ from place to place and time to time. This fact proves that there 
is a large conventional factor involved. But there is no group at any time or place which 
does not have some code of manners as, for example, with respect to proper ways of 
greeting other persons. The particular form a convention takes has nothing fixed and 
absolute about it. But the existence of some form of convention is not itself a convention. 
It is a uniform attendant of all social relationships. At the very least, it is the oil which 
prevents or reduces friction. 

It is possible, of course, for these social forms to become, as we say, "mere 
formalities." They may become merely outward show with no meaning behind them. But 
the avoidance of empty ritualistic forms of social inter course does not mean the rejection 
of every formal element. It rather indicates the need for development of forms of 
intercourse that are inherently appropriate to social situations. Visitors to some 
progressive schools are shocked by the lack of manners they come across. One who 
knows the situation better is aware that to some extent their absence is due to the eager 
interest of children to go on with what they sue doing. In their eagerness they may, for 



example, bump into each other and into visitors with no word of apology. One might say 
that this condition is better than a display of merely external punctilio accompanying 
intellectual and emotional lack of interest in schoolwork. But it also represents a failure 
in education, a failure to learn one of the most important lessons of life, that of mutual 
accommodation and adaptation. Education is going on in a one-sided way, for attitudes 
and habits are in process of formation that stand in the way of the future learning that 
springs from easy and ready contact and communication with others. 


Chapter 5 

The Nature of Freedom 

AT THE RISK Of repeating what has been often said by me I want to say something 
about the other side of the problem of social control, namely, the nature of freedom. The 
only freedom that is of enduring importance is freedom of intelligence, that is to say, 
freedom of observation and of judgment exercised in behalf of purposes that are 
intrinsically worth while. The commonest mistake made about freedom is, I think, to 
identify it with freedom of movement, or with the external or physical side of activity. 
Now, this external and physical side of activity cannot be separated from the internal side 
of activity; from freedom of thought, desire, and purpose. The Limitation that was put 
upon outward action by the fixed arrangements of the typical traditional schoolroom, 
with its fixed rows of desks and its military regimen of pupils who were permitted to 
move only at certain fixed signals, put a great restriction upon intellectual and moral 
freedom. Straitjacket and chain-game procedures had to be done away with if there was 
to be a chance for growth of individuals in the intellectual springs of freedom without 
which there is no assurance of genuine and continued normal growth. 

But the fact still remains that an increased measure of freedom of outer movement is a 
means, not an end. The educational problem is not solved when this aspect of freedom is 
obtained. Everything then depends, so far as education is concerned, upon what is done 
with this added liberty. What end does it serve? What consequences flow from it? Let me 
speak first of the advantages which reside potentially in increase of outward freedom. In 
the first place, without its existence it is practically impossible for a teacher to gain 
knowledge of the individuals with whom he is concerned. Enforced quiet and 
acquiescence prevent pupils from disclosing their real natures. They enforce artificial 
uniformity. They put seeming before being. They place a premium upon preserving the 
outward appearance of attention, decorum, and obedience. And everyone who is 
acquainted with schools in which this system prevailed well knows that thoughts, 
imaginations, desires, and sly activities ran their own unchecked course behind this 
facade. They were disclosed to the teacher only when some untoward act led to their 
detection. One has only to contrast this highly artificial situation with normal human 
relations outside the schoolroom, say in a well conducted home, to appreciate how fatal it 
is to the teacher's acquaintance with and understanding of the individuals who are, 
supposedly, being educated. Yet without this insight there is only an accidental chance 
that the material of study and the methods used in instruction will so come home to an 


individual that his development of mind and character is actually directed. There is a 
vicious circle. Mechanical uniformity of studies and methods creates a kind of uniform 
immobility and this reacts to perpetuate uniformity of studies and of recitations, while 
behind this enforced uniformity individual tendencies operate in irregular and more or 
less forbidden ways. 

The other important advantage of increased outward freedom is found in the very 
nature of the learning process. That the older methods set a premium upon passivity and 
receptivity has been pointed out. Physical quiescence puts a tremendous premium upon 
these traits. The only escape from them in the standardized school is an activity, which is 
irregular and perhaps disobedient. There cannot be complete quietude in a laboratory or 
workshop. The non-social character of the traditional school is seen in the fact that it 
erected silence into one of its prime virtues. There is, of course, such a thing as intense 
intellectual activity without overt bodily activity. But capacity for such intellectual 
activity marks a comparatively late achievement when it is continued for a long period. 
There should be brief intervals of time for quiet reflection pro- vided for even the young. 
But they are periods of genuine reflection only when they follow after times of more 
overt action and are used to organize what has been gained in periods of activity in which 
the hands and other parts of the body beside the brain are used. Freedom of movement is 
also important as a means of maintaining normal physical and mental health. We have 
still to learn from the example of the Greeks who saw clearly the relation between a 
sound body and a sound mind. But in all the respects mentioned freedom of outward 
action is a means to freedom of judgment and of power to carry deliberately chosen ends 
into execution. The amount of external freedom, which is needed, varies from individual 
to individual. It naturally tends to decrease with increasing maturity, though its complete 
absence prevents even a mature individual from having the contacts, which will provide 
him with new materials upon which his intelligence may exercise itself. The amount and 
the quality of this kind of free activity as a means of growth is a problem that must 
engage the thought of the educator at every stage of development. 

There can be no greater mistake, however, than to treat such freedom as an end in 
itself. It then tends to be destructive of the shared cooperative activities which are the 
normal source of order. But, on the other hand, it turns freedom which should be positive 
into something negative. For freedom from restriction, the negative side, is to be prized 
only as a means to a freedom which is power: power to frame purposes, to judge wisely, 
to evaluate desires by the consequences which will result from acting upon them; power 
to select and order means to carry chosen ends into operation. 

Natural impulses and desires constitute in any case the starting point. But there is no 
intellectual growth without some reconstruction, some remaking, of impulses and desires 
in the form in which they first show themselves. This remaking involves inhibition of 
impulse in its first estate. The alternative to externally imposed inhibition is inhibition 
through an individual's own reflection and judgment. The old phrase "Stop and think" is 
sound psychology. For thinking is stoppage of the immediate manifestation of impulse 
until that impulse has been brought into connection with other possible tendencies to 
action so that a more comprehensive and coherent plan of activity is formed. Some of the 



other tendencies to action lead to use of eye, ear, and hand to observe objective 
conditions; others result in recall of what has happened in the past. Thinking is thus a 
postponement of immediate action, while it effects internal control of impulse through a 
union of observation and memory, this union being the heart of reflection. What has been 
said explains the meaning of the well-worn phrase "self-control." The ideal aim of 
education is creation of power of self-control. But the mere removal of external control is 
no guarantee for the production of self-control. It is easy to jump out of the frying-pan 
into the fire. It is easy, in other words, to escape one form of external control only to find 
oneself in another and more dangerous form of external control. Impulses and desires that 
are not ordered by intelligence are under the control of accidental circumstances. It may 
be a loss rather than a gain to escape from the control of another person only to find one's 
conduct dictated by immediate whim and caprice; that is, at the mercy of impulses into 
whose formation intelligent judgment has not entered. A person whose conduct is 
controlled in this way has at most only the illusion of freedom. Actually forces over 
which he has no command direct him. 


Chapter 6 

The Meaning of Purpose 

IT IS, then, a sound instinct which identifies freedom with power to frame purposes 
and to execute or carry into effect purposes so framed. Such freedom is in turn identical 
with self-control; for the formation of purposes and the organization of means to execute 
them are the work of intelligence. Plate once defined a slave as the person who executes 
the purposes of another, and, as has just been said, a person is also a slave who is 
enslaved to his own blind desires. There is, I think, no point in the philosophy of 
progressive education which is sounder than its emphasis upon the importance of the 
participation of the learner in the formation of the purposes which direct his activities in 
the learning process, just as there is no defect in traditional education greater than its 
failure to secure the active cooperation of the pupil in construction of the purposes 
involved in his studying. But the meaning of purposes and ends is not self-evident and 
self-explanatory. The more their educational importance is emphasized, the more 
important it is to understand what a purpose is; how it arises and how it functions in 
experience. 

A genuine purpose always starts with an impulse. Obstruction of the immediate 
execution of an impulse converts it into a desire. Nevertheless neither impulse nor desire 
is itself a purpose. A purpose is an end-view. That is, it involves foresight of the 
consequences which will result from acting upon impulse. Foresight of consequences 
involves the operation of intelligence. It demands, in the first place, observation of 
objective conditions and circumstances. For impulse and desire produce consequences 
not by themselves alone but through their interaction or co- operation with surrounding 
conditions. The impulse for such a simple action as walking is executed only in active 
conjunction with the ground on which one stands. Under ordinary circumstances, we do 
not have to pay much attention to the ground. In a ticklish situation we have to observe 


very carefully just what the conditions are, as in climbing a steep and rough mountain 
where no trail has been laid out. Exercise of observation is, then, one condition of 
transformation of impulse into a purpose. As in the sign by a railway crossing, we have to 
stop, look, and listen. 

But observation alone is not enough. We have to understand the significance of what 
we see, hear, and touch. This significance consists of the consequences that will result 
when what is seen is acted upon. A baby may see the brightness of a dame and be 
attracted thereby to reach for it. The significance of the flame is then not its brightness 
but its power to burn, as the consequence that will result from touching it. We can be 
aware of consequences only because of previous experiences. In cases that are familiar 
because of many prior experiences we do not have to stop to remember just what those 
experiences were. A dame comes to signify light and heat without our having expressly 
to think of previous experiences of heat and burning. But in unfamiliar cases, we cannot 
tell just what the consequences of observed conditions will be unless we go over past 
experiences in our mind, unless we reflect upon them and by seeing what is similar in 
them to those now present, go on to form a judgment of what may be expected in the 
present situation. The formation of purposes is, then, a rather complex intellectual 
operation. It involves (1) observation of surrounding conditions; (2) knowledge of what 
has happened in similar situations in the past, a knowledge obtained partly by recollection 
and partly from the in- formation, advice, and warning of those who have had a wider 
experience; and (3) judgment which puts together what is observed and what is recalled 
to see what they signify. A purpose differs from an original impulse and desire through 
its translation into a plan and method of action based upon foresight of the consequences 
of acting under given observed conditions in a certain way. "If wishes were horses, 
beggars would ride." Desire for some thing may be intense. It may be so strong as to 
override estimation of the consequences that will follow acting upon it. Such occurrences 
do not provide the model for education. The crucial educational problem is that of pre 
curing the postponement of immediate action upon desire until observation and judgment 
have intervened. Unless I am mistaken, this point is definitely relevant to the conduct of 
progressive schools. Overemphasis upon activity as an end, instead of upon intelligent 
activity, leads to identification of freedom with immediate execution of impulses and 
desires. This identification is justified by a confusion of impulse with purpose; although, 
as has just been said, there is no purpose unless overt action is postponed until there is 
foresight of the consequences of carrying the impulse into execution-a foresight that is 
impossible without observation, information, and judgment. Mere foresight, even if it 
takes the form of accurate prediction, is not, of course, enough. The intellectual 
anticipation, the idea of consequences, must blend with desire and impulse to acquire 
moving force. It then gives direction to what otherwise is blind, while desire gives ideas 
impetus and momentum. An idea then becomes a plan in and for an activity to be carried 
out. Suppose a man has a desire to secure a new home, say by building a house. No 
matter how strong his desire, it cannot be directly executed. The man must form an idea 
of what kind of house he wants, including the number and arrangement of rooms, etc. He 
has to draw a plan, and have blue prints and specifications made. Ah this might be an idle 
amusement for spare time unless he also took stock of his resources. He must consider 
the relation of his funds and available credit to the execution of the plan. He has to 



investigate available sites, their price, their nearness to his place of business, to a 
congenial neighborhood, to school facilities, and so on and so on. All of the things 
reckoned with: his ability to pay, size and needs of family, possible locations, etc., etc., 
are objective facts. They are no part of the original desire. But they have to be viewed 
and judged in order that a desire may be converted into a purpose and a purpose into a 
plan of action. 

All of us have desires, all at least who have not become so pathological that they are 
completely apathetic. These desires are the ultimate moving springs of action. A 
professional businessman wishes to succeed in his career; a general wishes to win the 
battle; a parent to have a comfortable home for his family, and to educate his children, 
and so on indefinitely. The intensity of the desire measures the strength of the efforts that 
will be put forth. But the wishes are empty castles in the air unless they are translated into 
the means by which they may be realized. The question of how soon of means takes the 
place of a projected imaginative end, and, since means are objective, they have to be 
studied and understood if a genuine purpose is to be formed. 

Traditional education tended to ignore the importance personal impulse and desire as 
moving springs. But this is no reason why progressive education should identify impulse 
and desire with purpose and thereby pass lightly over the need for careful observation, for 
wide range of information, and for judgment if students are to share in the formation of 
the purposes which activate them. In an educational scheme, the occurrence of a desire 
and impulse is not the final end. It is an occasion and a demand for the formation of a 
plan and method of activity. Such a plan, to repeat, can be formed only by study of 
conditions and by sewing all relevant information. 

The teacher's business is to see that the occasion is taken advantage of. Since freedom 
resides in the operations of intelligent observation and judgment by which a purpose is 
developed, guidance given by the teacher to the exercise of the pupils' intelligence is an 
aid to freedom, not a restriction upon it. Sometimes teachers seem to be afraid even to 
make suggestions to the members of a group as to what they should do. I have heard of 
cases in which children are surrounded with objects and materials and then left entirely to 
themselves, the teacher being loath to suggest even what might be done with the 
materials lest freedom be infringed upon. Why, then, even supply materials, since they 
are a source of some suggestion or other? But what is more important is that the 
suggestion upon which pupils act must in any case come from some- where. It is 
impossible to understand why a suggestion from one who has a larger experience and a 
wider horizon should not be at least as valid as a suggestion arising from some more or 
less accidental source. 

It is possible of course to abuse the office, and to force the activity of the young into 
channels which express the teacher's purpose rather than that of the pupils. But the way to 
avoid this danger is not for the adult to withdraw entirely. The way is, first, for the 
teacher to be intelligently aware of the capacities, needs, and past experiences of those 
under instruction, and, secondly, to allow the suggestion made to develop into a plan and 
project by means of the further suggestions contributed and organized into a whole by the 



members of the group. The plan, in other words, is a co-operative enterprise, not a 
dictation. The teacher's suggestion is wt a mold for a cast-iron result but is a starting point 
to be developed into a plan through contributions from the experience of all engaged in 
the learning process. The development occurs through reciprocal give-and-take, the 
teacher taking but not being afraid also to give. The essential point is that the purpose 
grow and take shape through the process of social intelligence. 


Chapter 7 

Progressive Organization of Subject Matter 

ALLUSION HAS been made in passing a number of times to objective conditions 
involved in experience and to their function in promoting or failing to promote the 
enriched growth of further experience. By implication, these objective conditions, 
whether those of observation, of memory, of information procured from others, or of 
imagination, have been identified with the subject-matter of study and learning; or, 
speaking more generally, with the stuff of the course of study. Nothing, however, has 
been said explicitly so far about subject-matter as such. That topic will now be discussed. 
One consideration stands out clearly when education is conceived in terms of experience. 
Anything which can be called a study, whether arithmetic, history, geography, or one of 
the natural sciences, must be derived from materials which at the outset fall within the 
scope of ordinary life-experience. In this respect the newer education contrasts sharply 
with procedures which start with facts and truths that are outside the range of the 
experience of those thought, and which, therefore, have the problem of discovering ways 
and means of bringing them within experience. Undoubtedly one chief cause for the great 
success of newer methods in early elementary education has been its observance of the 
contrary principle. 

But finding the material for learning within experience is only the first step. The next 
step is the progressive development of what is already experienced into a fuller and richer 
and also more organized form, a form that gradually approximates that in which subject- 
matter is presented to the skilled, mature person. That this change is possible without 
departing from the organic connection of education with experience is shown by the fact 
that this change takes place outside of the school and apart from formal education. The 
infant, for example, begins with an environment of objects that is very restricted in space 
and time. That environment steadily expands by the momentum inherent in experience 
itself without aid from scholastic instruction. As the infant learns to reach, creep, walk, 
and talk, the intrinsic subject-matter of its experience widens and deepens. It comes into 
connection with new objects and events, which call out new powers, while the exercise of 
these powers refines and enlarges the content of its experience. Life-space and life- 
duration’s are expanded. The environment, the world of experience, constantly grows 
larger and, so to speak, thicker. The educator who receives the child at the end of this 
period has to find ways for doing consciously and deliberately what "nature" 
accomplishes in the earlier years. 


It is hardly necessary to insist upon the first of the two conditions which have been 
specified. It is a cardinal precept of the newer school of education that the beginning of 
instruction shall be made with the experience learners already have; that this experience 
and the capacities that have been developed during its course provide the starting point 
for ah further learning. I am not so sure that the other condition, that of orderly 
development toward expansion and organization of subject-matter through growth of 
experience, receives as much attention. Yet the principle of continuity of educative 
experience requites that equal thought and attention be given to solution of this aspect of 
the educational problem. Undoubtedly this phase of the problem is more difficult than the 
other. Those who deal with the pre-school child, with the kindergarten child, and with the 
boy and girl of the early primary years do not have much difficulty in determining the 
range of past experience or in finding activities that connect in vital ways with it. With 
older children both factors of the problem offer increased difficulties to the educator. It is 
harder to find out the background of the experience of individuals and harder to find out 
just how the subject- matters already contained in that experience shall be directed so as 
to lead out to larger and better organized fields. 

It is a mistake to suppose that the principle of the leading on of experience to 
something different is adequately satisfied simply by giving pupils some new experiences 
any more than it is by seeing to it that they have greater still and ease in dealing with 
things with which they are already familiar. It is also essential that the new objects and 
events be related intellectually to those of earlier experiences, and this means that there 
be some advance made in conscious articulation of facts and ideas. It thus becomes the 
office of the educator to select those things within the range of existing experience that 
have the promise and potentiality of presenting new problems which by stimulating new 
ways of observation and judgment will expand the area of further experience He must 
constantly regard what is already won not as a fixed possession but as an agency and 
instrumentality for opening new fields which make new demands upon existing powers 
of observation and of intelligent use of memory. Connectedness in growth must be his 
constant watchword. The educator more than the member of any other profession is 
concerned to have a long look ahead. The physician may feel his job done when he has 
restored a patient to health. He has undoubtedly the obligation of advising him bow to 
live so as to avoid similar troubles in the future. But, after all, the conduct of his life is 
his own affair, not the physician's; and what is more important for the present point is that 
as far as the physician does occupy himself with instruction and advice as to the future of 
his patient he takes upon himself the function of an educator. The lawyer is occupied with 
winning a suit for his client or getting the latter out of some complication into which he 
has got himself. If it goes beyond the case presented to him he too becomes an educator. 
The educator by the very nature of his work is obliged to see his present work in terms of 
what it accomplishes, or fails to accomplish, for a future whose objects are linked with 
those of the present. 

Here, again, the problem for the progressive educator is more difficult than for the 
teacher in the traditional school. The latter had indeed to look ahead. But unless his 
personality and enthusiasm took him beyond the limits that hedged in the traditional 



school, he could content himself with thinking of the next examination period or the 
promotion to the next class. He could envisage the future in terms of factors that lay 
within the requirements of the school system as that conventionally existed. There is 
incumbent upon the teacher who links education and actual experience together a more 
serious and a harder business. He must be aware of the potentialities for leading students 
into new fields which belong to experiences already had, and must use this knowledge as 
his criterion for selection and arrangement of the conditions that influence their present 
experience. 

Because the studies of the traditional school consisted of subject-matter that was 
selected and arranged on the basis of the judgment of adults as to what would be useful 
for the young sometime in the future, the material to be learned was settled upon outside 
the present life -experience of the learner. In consequence, it had to do with the past; it 
was such as had proved useful to men in past ages. By reaction to an opposite extreme, as 
unfortunate as it was probably natural under the circumstances, the sound idea that 
education should derive its materials from present experience and should enable the 
learner to cope with the problems of the present and future has often been converted into 
the idea that progressive schools can to a very large extent ignore the past. If the present 
could be cut off from the past, this conclusion would be sound. But the achievements of 
the past provide the only means at command for understanding the present. Just as the 
individual has to draw in memory upon his own past to understand the conditions in 
which he individually finds himself, so the issues and problems of present social life are 
in such intimate and direct connection with the past that students cannot be prepared to 
understand either these problems or the best way of dealing with them without delving 
into their roots in the past. In other words, the sound principle that the objectives of 
learning are in the future and its immediate materials are in present experience can be 
carried into effect only in the degree that present experience is stretched, as it were, 
backward. It can expand into the future only as it is also enlarged to take in the past. 

If time permitted, discussion of the political and economic issues which the present 
generation will be compelled to face in the future would render this general statement 
definite and concrete. The nature of the issues cannot be understood save as we know 
how they came about. The institutions and customs that exist in the present and that give 
rise to present social ills and dislocations did not arise overnight. They have a long 
history behind them. Attempt to deal with them simply on the basis of what is obvious in 
the present is bound to result in adoption of superficial measures which in the end will 
only render existing problems more acute and more difficult to solve. Policies framed 
simply upon the ground of knowledge of the present cut off from the past is the 
counterpart of heedless carelessness in individual conduct. The way out of scholastic 
systems that made the past an end in itself is to make acquaintance with the past a means 
of understanding the present. Until this problem is worked out, the present clash of 
educational ideas and practices will continue. On the one hand, there win be reactionaries 
that claim that the main, if not the sole, business of education is transmission of the 
cultural heritage. On the other hand, there will be those who hold that we should ignore 
the past and deal only with the present and future. 



That up to the present time the weakest point in progressive schools is in the matter of 
selection and organization of intellectual subject-matter isr I think, inevitable under the 
circumstances. It is as inevitable as it is right and proper that they should break loose 
from the cut and dried material which formed the staple of the old education, In addition, 
the field of experience is very wide and it varies in its contents from place to place and 
from time t, time. A single course of studies for ah progressive schools is out of the 
question; it would mean abandoning the fundamental principle of connection with life- 
experiences. Moreover, progressive schools are new. They have had hardly more than , 
generation in which to develop. A certain amount of uncertainty and of laxity in choice 
and organization of subject-matter is, therefore, what was to be expected. It is no ground 
for fundamental criticism or complaint. 

It is a ground for legitimate criticism, however, when the ongoing movement of 
progressive education fails to recognize that the problem of selection and organization ~ 
subject-matter for study and learning is fundamental. Improvisation that takes advantage 
of special occasions prevents teaching and learning from being stereotyped and dead. 
But the basic material of study cannot be picked up in a cursory manner. Occasions 
which are not and cannot be foreseen are bound to arise wherever there is intellectual 
freedom. They should be utilized. But there is a decided difference between using them in 
the development of a continuing line of activity and trusting to them to provide the chief 
material of learning. 

Unless a given experience leads out into a held previously unfamiliar no problems 
arise, while problems are the stimulus to thinking. That the conditions found in present 
experience should be used as sources of problems is a characteristic which differentiates 
education based upon experience from traditional education. For in the latter, problems 
were set from outside. Nonetheless, growth depends upon the presence of difficulty to be 
overcome by the exercise of intelligence. Once more, it is part of the educator's 
responsibility to see equally to two things: First, that the problem grows out of the 
conditions of the experience being had in the present and that it is within the range of the 
capacity of students; and, secondly, that it is such that it arouses in the learner an active 
quest for information and for production of new ideas. The new facts and new ideas thus 
obtained become the ground for further experiences in which new problems me 
presented. The process is a continuous spiral The inescapable linkage of the present with 
the past is a principle whose application is not restricted to a study of history. Take 
natural science, for example. Contemporary social Life is what it is in very large measure 
because of the results of application of physical science. The experience of every child 
and youth, in the country and the city, is what it is in its present actuality became of 
appliances which utilize electricity, heat, and chemical processes. A child does not eat a 
meal that does not involve in its preparation and assimilation chemical and physiological 
principles. He does not read by artificial light or take a ride in a motor car or on a train 
without coming into contact with operations and processes which science has 
engendered. 

It is a sound educational principle that students should be introduced to scientific 
subject-matter and be initiated into its facts and laws through acquaintance with everyday 



social applications. Adherence to this method is Mt only the most direct avenue to 
understanding of science itself but as the pupils grow more mature it is also the surest 
road to the understanding of the economic and industrial problems of present society. For 
they are the products to a very large extent of the application of science in production and 
distribution of commodities and services, while the latter processes are the most 
important factor in determining the present relations of human beings and social groups 
to one another. It is absurd, then, to argue that processes similar to those studied in 
laboratories and institutes of research are not a part of the daily life- experience of the 
young and hence do not come within the scope of education based upon experience. That 
the immature cannot study scientific facts and principles in the way in which mature 
experts study them goes without saying. But this fact, instead of exempting the educator 
from responsibility for using present experiences so that learners may gradually be led, 
through extraction of facts and laws, to experience of a scientific order, sets one of his 
main problems. 

For if it is true that existing experience in detail and also on a wide scale is what it is 
because of the application of science, first, to processes of production and distribution of 
goods and services, and then to the relations which human beings sustain socially to one 
another, it is impossible to obtain an understanding of present social forces (without 
which they cannot be mastered and directed) apart from an education which leads 
learners into knowledge of the very same facts and principles which in their final 
organization constitute the sciences. Nor does the importance of the principle that 
learners should be led to acquaintance with scientific subject-matter cease with the 
insight thereby given into present social issues. The methods of science also point the 
way to the measures and policies by means of which a better social order can be brought 
into existence. The applications of science which have produced in large measure the 
social conditions which now exist do not exhaust the possible field of their application. 
For so far science has been applied more or less casually and under the influence of ends, 
such as private advantage and power, which are a heritage from the institutions of a pre- 
scientific age. 

We are told almost daily and from many sources that it is impossible for human beings 
to direct their common life intelligently. We are told, on one hand, that the complexity of 
human relations, domestic and international, and on the other hand, the fact that human 
beings are so largely creatures of emotion and habit, make impossible large-scale social 
planning and direction by intelligence. This view would be more credible if any 
systematic effort, beginning with early education and carried on through the continuous 
study and learning of the young, had ever been undertaken with a view to making the 
method of intelligence, exemplified in science, supreme in education. There is nothing in 
the inherent nature of habit that prevents intelligent method from becoming itself 
habitual; and there is nothing in the nature of emotion to prevent the development of 
intense emotional allegiance to the method. 

The case of science is here employed as an illustration of progressive selection of 
subject-matter resident in present experience towards organization: an organization which 
is free, not externally imposed, because it is in accord with the growth of experience 



itself. The utilization of subject-matter found in the present life-experience of the learner 
towards science is perhaps the best illustration that can be found of the basic principle of 
using existing experience as the means of carrying learners on to a wider, more refined, 
and better organized environing world, physical and human, than is found in the 
experiences from which educative growth sets out. Hogben's recent work Mathematics 
for the Million, shows how mathematics, if it is treated as a mirror of civilization and as 
a main agency in its progress, can contribute to- the desired goal its surely as can the 
physical sciences. The underlying ideal in any case is that of progressive organization of 
knowledge. It is with reference to organization of knowledge that we are likely to find 
Either-Or philosophies most acutely active. In practice, if not in so many words, it is 
often held that since traditional education rested upon a conception of organization of 
knowledge that was almost completely contemptuous of living present experience, 
therefore education based upon living experience should be contemptuous of the 
organization of facts and ideas. 

When a moment ago I called this organization an ideal, I meant, on the negative side, 
that the educator cannot start with knowledge already organized and proceed to lade it out 
in doses. But as an ideal the active process of organization facts and ideas is an ever- 
present educational process. No experience is educative that does not tend both to 
knowledge of more facts and entertaining of more ideas and to a better, a more orderly, 
arrangement of them. It is not true that organization is a principle foreign to experience. 
Otherwise experience would be so dispersive as to be chaotic. The experience of young 
children centers about persons and the home. Disturbance of the normal order of 
relationships in the family is now known by psychiatrists to be a fertile source of later 
mental, and : emotional troubles— a fact which testifies to the reality of this kind of 
organization. One of the great advances in early school education, in the kindergarten and 
early grades, is that it preserves the social and human center of the organization of 
experience, instead of the older violent shift of the center of gravity. But one of the 
outstanding problems of education, as of music, is modulation. In the case of education, 
modulation means movement from a social and human center toward a more objective 
intellectual scheme of organization, always hearing in mind, however, that intellectual 
organization is not an end in itself but is the means by which social relations, distinctively 
human ties and bonds, may be understood and more intelligently ordered. 

When education is based in theory and practice upon experience, it goes without 
saying that the organized subject-matter of the adult and the specialist cannot provide the 
starting point. Nevertheless, it represents the goal toward which education should 
continuously move. It is hardly necessary to say that one of the most fundamental 
principles of the scientific organization of knowledge is principle of cause-and-effect. 
The Hay in which this principle is grasped and formulated by the scientific specialist is 
certainly very different from the way in which can be approached in the experience of the 
young. But neither the relation nor grasp of its meaning is foreign to the experience of 
even the young child. When a child two or three years of age learns not to approach a 
flame too closely and yet to draw near enough a stove to get its warmth he is grasping 
and using the causal relation. There is no intelligent activity that does not conform to the 



requirements of the relation, and it is intelligent in the degree in which it is not only 
conformed to but consciously borne in mind. 

In the earlier forms of experience the causal relation does not offer itself in the 
abstract but in the form of the relation of means employed to ends attained; of the relation 
of means and consequences. Growth in judgment and understanding is essentially growth 
in ability to form purposes and to select and arrange means for their realization. The most 
elementary experiences of the young are filled with cases of the means-consequence 
relation. There is not a meal cooked nor a source of illumination employed that does not 
exemplify this relation. The trouble with education is not the absence of situations in 
which the causal relation is exemplified in the relation of means and consequences. 
Failure to utilize the situations so as to lead the learner on to grasp the relation in the 
given cases of experience is, however, only too common. The logician gives the names 
"analysis and synthesis" to the operations by which means are selected and organized in 
relation to a purpose. 

This principle determines the ultimate foundation for the utilization of activities in 
school. Nothing can be more absurd educationally than to make a plea for a variety of 
active occupations in the school while decrying the need for progressive organization of 
information and ideas. Intelligent activity is distinguished from aimless activity by the 
fact that it involves selection of means-analysis-out of the variety of conditions that are 
present, and their arrangement-synthesis-to reach an intended aim or purpose. That the 
more immature the learner is, the simpler must be the ends held in view and the more 
rudimentary the means employed, is obvious. But the principle of organization of activity 
in terms of some perception of the relation of consequences to means applies even with 
the very young. Otherwise an activity ceases to be educative because it is blind. With 
increased maturity, the problem of interrelation of means becomes more urgent. In the 
degree in which intelligent observation is transferred from the relation of means to ends 
to the more complex question of the relation of means to one another, the idea of cause 
and effect becomes prominent and explicit. The final justification of shops, kitchens, and 
so on in the school is not just that they afford opportunity for activity, but that they 
provide opportunity for the kind of activity or for the acquisition of mechanical skills 
which leads students to attend to the relation of means and ends, and then to 
consideration of the way things interact with one another to produce definite effects. It is 
the same in principle as the ground for laboratories in scientific research. 

Unless the problem of intellectual organization can be worked out on the ground of 
experience, reaction is sure to occur toward externally imposed methods of organization. 
There pre signs of this reaction already in evidence. We are told that our schools, old and 
new, are failing in the main task. They do not develop, it is said, the capacity for critical 
discrimination and the ability to reason. The ability to think is smothered, we are told, by 
accumulation of miscellaneous ill-digested information, and by the attempt to acquire 
forms of skill which will be immediately useful in the business and commercial world. 
We are told that these evils spring from the influence of science and from the 
magnification of present requirements at the expense of the tested cultural heritage from 
the past. It is argued that science and its method must be subordinated; that we must 



return to the logic of ultimate first principles expressed in the logic of Aristotle and St. 
Thomas, in order that the young may have sure anchorage in their intellectual and moral 
life, and not be at the mercy of every passing breeze that blows. 

If the method of science had ever been consistently and continuously applied 
throughout the day-by-day work of the school in all subjects, I should be more impressed 
by this emotional appeal than I am. I see at bottom but two alternatives between which 
education must choose if it is not to drift aimlessly. One of them is expressed by the 
attempt to induce educators to return to the intellectual methods and ideals that arose 
centuries before scientific method was developed. The appeal may be temporarily 
successful in a period when general insecurity, emotional and intellectual as well as 
economic, is rife. For under these conditions the desire to lean on fixed authority is 
active. Nevertheless, it is so out of touch with all the conditions of modern life that I 
believe it is folly to seek salvation in this direction. The other alternative is systematic 
utilization of scientific method as the pattern and ideal of intelligent exploration and 
exploitation of the potentialities inherent in experience. 

The problem involved comes home with peculiar force to progressive schools. Failure 
to give constant attention to development of the intellectual content of experiences and to 
obtain ever-increasing organization of facts and ideas may in the end merely strengthen 
the tendency towards a reactionary return to intellectual and moral authoritarianism. The 
present is not the time nor place for acquisition upon scientific method. But certain 
features of it are so closely connected with any educational scheme based upon 
experience that they should be noted. In the first place, the experimental method of 
science attaches more importance, not less, to ideas as ideas than do other methods. There 
is no such thing as experiment in the scientific sense unless action is directed by some 
lead- idea. The fact that the ideas employed are hypotheses, not final truths, is the reason. 
Why ideas are more jealously guarded and tested in science than anywhere else. The 
moment they are taken to be first truths in themselves there ceases to be any reason for 
scrupulous examination of them. As fixed truths they must he accepted and that is the end 
of the matter. But as hypotheses, they must be continuously tested and revised, a 
requirement that demands they be accurately formulated. 

In the second place, ideas or hypotheses are tested by the consequences, which they 
produce when they are acted upon. This fact means that the consequences of action must 
be carefully and discriminatingly observed Activity that is not checked by observation of 
what follows from it may be temporarily enjoyed. But intellectually it leads nowhere. It 
does not provide knowledge about the situations in which action occurs nor does it lead 
to clarification and expansion of ideas. 

In the third place, the method of intelligence manifested in the experimental method 
demands keeping track of ideas, activities, and observed consequences. Keeping track is 
a matter of reflective review and summarizing, in which there is both discrimination and 
record of the significant features of a developing experience. To reflect is to look back 
over what has been done so as to extract the net meanings, which are the capital stock for 



intelligent dealing with further experiences. It is the heart of intellectual organization and 
of the disciplined mind. 

I have been forced to speak in general and often abstract language. But what has been 
said is organically connected with the requirement that experiences in order to be 
educative must lead out into an expanding world of subject-matter, 1 subject-matter of 
facts or information and of ideas. This condition is satisfied only as the educator views 
teaching and learning as a continuous process of reconstruction of experience. This 
condition in turn can be satisfied only as the educator has a long look ahead, and views 
every present experience as a moving force in influencing what future experiences will 
be. I am aware that the emphasis I have placed upon scientific method may be 
misleading, for it may result only in calling up the special technique of laboratory 
research as that is conducted by specialists. But the meaning of the emphasis placed upon 
scientific method has little to do with specialized techniques. It means that scientific 
method is the only authentic means at our command for getting at the significance of our 
everyday experiences of the world in which we live. It means that scientific method 
provides a working pattern of the way in which and the conditions under which 
experiences are used to lead ever onward and outward. Adaptation of the method to 
individuals of various degrees of maturity is a problem for the educator, and the constant 
factors in the problem are the formation of ideas, acting upon ideas, observation of the 
conditions, which result, and organization of facts and ideas for future use. Neither the 
ideas, nor the activities, nor the observations, the organization are the same for a person 
six years old as they are for one twelve or eighteen years old, to say nothing of the adult 
scientist. But at every level there is an expanding development of experience if 
experience is educative in effect. Consequently, whatever the level of experience, we 
have no choice but either to operate in accord with the pattern it provides or else to 
neglect the place of intelligence in the development and control of a living and moving 
experience. 


Chapter 8 

Experience-The Means and Goal of Education 

IN WHAT I HAVE SAID I have taken for granted the sound- ness of the principle 
that education in order to accomplish its ends both for the individual learner and for 
society must be based upon experience— which is always the actual life-experience of 
some individual. I have not argued for the acceptance of this principle nor attempted to 
justify it. Conservatives as well as radicals in education are profoundly discontented with 
the present educational situation taken as a whole. There is at least this much agreement 
among intelligent persons of both schools of educational thought. The educational system 
must move one way or another, either backward to the intellectual and moral standards of 
a pre-scientific age or forward to ever greater utilization of scientific method in the 
development of the possibilities of growing, expanding experience. I have but 
endeavored to point out some of the conditions, which must be satisfactorily fulfilled if 
education takes the latter course. 


For I am so confident of the potentialities of education when it is treated as 
intelligently directed development of the possibilities inherent in ordinary experience that 
I do not feel it necessary to criticize here the other route nor to advance arguments in 
favor of taking the route of experience. The only ground for anticipating failure in taking 
this path resides to my mind in the danger that experience and the experimental method 
will not be adequately conceived. There is no discipline in the world so severe as the 
discipline of experience subjected to the tests of intelligent development and direction. 
Hence the only ground I can see for even a temporary reaction against the standards, 
aims, and methods of the newer education is the failure of educators who professedly 
adopt them to be faithful to them in practice. As I have emphasized more than once, the 
road of the new education is not an easier one to follow than the old road but n more 
strenuous and difficult one. It will remain so until it has attained its majority and that 
attainment will require many years of serious co-operative work on the part of its 
adherents. The greatest danger that attends its future is, I believe, the idea that it is an 
easy way to follow, so easy that its course may be improvised, if not in an impromptu 
fashion, at least almost from day to day or from week to week. It is for this reason that 
instead of extolling its principles, I have confined myself to showing certain conditions 
which must be fulfilled if it is to have the successful career which by right belongs to it. 

I have used frequently in what precedes the words "progressive" and '"new" education. 
I do not wish to close, however, without recording my firm belief that the fundamental 
issue is not of new versus old education nor of progressive against traditional education 
but a question of what anything whatever must be to be worthy of the name education. I 
am not, I hope and believe, in favor of any ends or any methods simply because the name 
progressive may be applied to them. The basic question concerns the nature of education 
with no qualifying adjectives prefixed. What we want and need is education pure and 
simple, and we shall make surer and faster progress when we devote ourselves to finding 
out just what education is and what conditions have to be satisfied in order that education 
may be a reality and not a name or a slogan. It is for this reason alone that I have 
emphasized the need for a sound philosophy of experience. 


END 

